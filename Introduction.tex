\chapter{Introduction}

The idea is to develop models and methods
for improving the service offered by insurance agents,
helping them arrive to the accident sites sooner,
and determining the number of adjusters required
to perform the service within the desired standards.

\section{Problem Statement}
Una compañia de seguros automovilisticos,
que cuenta con un grupo de operacion en monterrey N.L.,
se enfrenta a la inquietud constante de
ubicar a sus agentes aseguradores (ajustadores),
frente a diferentes escenarios que se presentan cotidianamente,
ya que su objevio,
es llegar a los lugares donde ocurra un incidente,
lo mas pronto posible.
Este es un problema de un sistema de emergencia,
en el cual se cubre todo el territorio de estudio,
con un numero fijo de ajustadores,
en el que se desea determinar
cuales deben ser las ubicaciones de los ajustadores
para asi reducir el tiempo promedio de arrivo a un incidente.

\section{Background}
Este es un problema practico que no ha sido abordado,
ya que este sistema de emergencia,
no se presenta en muchos paises,
sin embargo, hay sistemas similares,
tal es el caso de ambulancias, bomberos, patrullas,
para los cuales existen una gran cantidad de articulos al respecto,
pero las caracteristicas tratadas en ellos,
difieren de nuestro problema practico,
por los siguientes factores
tipo de espacio,
radio cobertura,
multiperiodo,
tiempo de servicio,
tipo de llegadas

Berman propone un problema de localizacion para sistemas de emergencia,
y lo llama The Queue Stochastic p-Median Problem.

\section{Motivation}
When a car accident occurs,
traffic congestion starts to pile up.
This is because
customers are not allowed to move their vehicles until the adjuster arrives.
The adjuster must record and determine the causes of the accident,
in order to move the car from the accident area
and restore the flow.

\section{Objectives}
The aim of this work is to support decision-making,
regardin location of insure agents to attend car wrecks.
The main goal is
to determine the optimal bases (locations)
for placing the insurance company adjusters,
so as to minimize
the average or maximum response time
from customer calls
when accidents occur.

\section{Organization}

