\section{Proposed Metaheuristic}
The Scatter Search (SS)
is a population-based metaheuristic
whose original ideas are due to Glover \cite{glover1977heuristics}
as a metaheuristic for integer programming.
It is based on diversifying the search
through the solution space.
It operates on a set of solutions,
named the Reference Set,
formed by good and diverse solutions of the main population.
These solutions are combined
with the aim of generate
new solutions with better fitness,
while maintaining diversity.
The General Scheme can be seen in Figure \ref{fig:SS_GS}.
\begin{figure}
  \centering
  \begin{tikzpicture} [
      auto,
      cloud/.style = { ellipse,
        draw=blue,
        fill=blue!20,
        node distance=3cm,
        minimum height=2em },
      decision/.style = { diamond,
        draw=blue,
        thick,
        fill=blue!20,
        text width=5em,
        text badly centered,
        inner sep=1pt,
        rounded corners },
      block/.style    = { rectangle,
        draw=blue,
        thick, 
        fill=blue!20,
        text width=10em,
        text centered,
        rounded corners,
        minimum height=2em },
      line/.style     = { draw, thick, ->, shorten >=2pt },
    ]
    % Define nodes in a matrix
    \matrix [column sep=5mm, row sep=10mm] {
      & \node [block] (init) {Diversification Generation};
      & &
      \\
      & \node (null1) {};
      & \node [block] (update) {\textsf{Update Reference Set}};
      & 
      \\
      & \node [cloud] (improve) {\textsf{Improvement}};
      & \node [decision] (newsol) {\textsf{Is there a new solution?}};
      & \node [cloud] (stop) {\textsf{Stop}};
      \\
      & \node [block] (combin) {\textsf{Solution Combination}};
      & \node [cloud] (subset) {\textsf{Subset Generation}};
      &
      \\
    };
    % connect all nodes defined above
    \begin{scope} [every path/.style=line]
      \path (init)     |-    (update);
      \path (update)   --    (newsol);
      \path (newsol)   --    node {no} (stop);
      \path (newsol)   --    node {yes} (subset);
      \path (subset)   --    (combin);
      \path (combin)   --    (improve);
      \path (improve)  --    (null1);
    \end{scope}
  \end{tikzpicture}
  \caption{Scatter Search General Scheme}
  \label{fig:SS_GS}
\end{figure}


\subsection{Diversity of a Solution}
% La diversidad de mide respecto al conjunto de referencia actual
The diversity of a solution,
is measured relative to the Reference Set;
% Se usa el perfect matching para calcular el costo
the perfect matching
is evaluated
with each one of the solutions in the Reference Set,
where the cost consist in
the sum of the distances
from each server of the given solution
and their match in the solution of the Reference Set.
% Se utiliza el minimo costo como medida de diversidad
The minimum cost perfect matching
is used as diversity value.
%\todo{Explicar como se calcula la diversidad}

\subsection{Scatter Search}
Scatter-Search metaheuristic
with a dynamic reference set
and a path relinking
as a combination method
was implemented.
We opted for a dynamic reference set
because the number of solutions generated
in each combination is very large.
To enhance the solutions
two improvement methods are included,
one proposed by Berman \cite{berman1987stochastic},
and a local search method
proposed in this research.
%\todo[inline]{Agregar background de Scatter Search}

The pseudo code
of the overall proposed scatter search metaheuristic
is shown in Algorithm \ref{alg:SSInitial}.

The initial step
is to generate the initial population
which is stored in the Reference Set.
This is depicted in Algorithm \ref{alg:SSInitial}.
The cardinality of the reference set is \textit{2b}
(for a given \textit{b}),
consisting of the bests \textit{b} solutions
by objective function value
and the \textit{b} most diverse solutions.
%%Pendiente revisar si no reemplaza definicion de diversidad previa
To measure the ``distance''
between any two solutions,
we used the value of the minimum perfect matching
between the corresponding center sets
of the two solutions.
That means that
a higher cost
indicates that
the solutions are distant.

\begin{algorithm}
  \caption{Scatter Search Initial Phase} \label{alg:SSInitial}
  \begin{algorithmic}
    \Procedure{Initial Phase}{} 
    \State $Iteration \gets 0$
    \State $IterationsUnchanged \gets 0$
    \State $RefSet \gets EmptySet()$
    \Repeat
    \State $Sols \gets DiversificationGenerator()$
    \State $Improvement(Sols)$
    \Comment {either of the two methods proposed}
    \State $RefSet.Update(Sols)$
    \State $Iteration \gets Iteration + 1$
    \State $IterationsUnchanged \gets IterationsUnchanged + 1$
    \If {$HasNewSolutions(RefSet)$}
    \State $IterationsUnchanged \gets 0$
    \EndIf
    \Until {$IterationsUnchanged \geq MaxIterUnchanged$
      \textbf{or} $Iteration \geq MaxIter$}
    \EndProcedure
  \end{algorithmic}
\end{algorithm}


In Algorithm \ref{alg:SSPhase} (see below),
each pair of solutions
is combined using path relinking procedure.
Because
the Reference Set is dynamic
some pairs are not be combined.
This is due the fact
that some solutions
are removed from the set
before trying to combine them.
\todo{Aclarar donde se manda llamar algoritmo 1}
\begin{algorithm}
  \caption{General Scheme of the Scatter Search} \label{alg:SSPhase}
  \begin{algorithmic}
    \Procedure{Scatter Search Phase}{$RefSet$}
    \Repeat
    \State $G_{X} \gets SubsetGeneration(RefSet)$
    \ForAll {$X \in G_X$}
    \If {$X \subset RefSet$}
    \Comment {Some element of $X$ may have been removed}
    \State $C_{X} = SolutionCombination(X)$
    \ForAll{$Sol \in C_X$}
    \State $Improvement(Sol)$
    \EndFor
    \State $RefSet.Update(C_{X})$
    \EndIf \EndFor
    \Until {$NoNewSolutions(RefSet)$}
    \EndProcedure
  \end{algorithmic}
\end{algorithm}

