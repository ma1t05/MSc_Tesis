\section{Proposed Metaheuristic}
The Scatter Search (SS)
is an evolutionary approach
introduced by Glover \cite{glover1977heuristics}
as a metaheuristic for integer programming.
It is based on diversifying the search
through the solution space.
It operates on a set of solutions,
named the Reference Set,
formed by good and diverse solutions of the main population.
These solutions are combined
with the aim of generate
new solutions with better fitness,
while maintaining diversity.
The General Scheme can be seen in Figure \ref{fig:SS_GS}.
\begin{figure}
  \centering
  \begin{tikzpicture} [
      auto,
      cloud/.style = { ellipse,
        draw=blue,
        fill=blue!20,
        node distance=3cm,
        minimum height=2em },
      decision/.style = { diamond,
        draw=blue,
        thick,
        fill=blue!20,
        text width=5em,
        text badly centered,
        inner sep=1pt,
        rounded corners },
      block/.style    = { rectangle,
        draw=blue,
        thick, 
        fill=blue!20,
        text width=10em,
        text centered,
        rounded corners,
        minimum height=2em },
      line/.style     = { draw, thick, ->, shorten >=2pt },
    ]
    % Define nodes in a matrix
    \matrix [column sep=5mm, row sep=10mm] {
      & \node [block] (init) {Diversification Generation};
      & &
      \\
      & \node (null1) {};
      & \node [block] (update) {\textsf{Update Reference Set}};
      & 
      \\
      & \node [cloud] (improve) {\textsf{Improvement}};
      & \node [decision] (newsol) {\textsf{Is there a new solution?}};
      & \node [cloud] (stop) {\textsf{Stop}};
      \\
      & \node [block] (combin) {\textsf{Solution Combination}};
      & \node [cloud] (subset) {\textsf{Subset Generation}};
      &
      \\
    };
    % connect all nodes defined above
    \begin{scope} [every path/.style=line]
      \path (init)     |-    (update);
      \path (update)   --    (newsol);
      \path (newsol)   --    node {no} (stop);
      \path (newsol)   --    node {yes} (subset);
      \path (subset)   --    (combin);
      \path (combin)   --    (improve);
      \path (improve)  --    (null1);
    \end{scope}
  \end{tikzpicture}
  \caption{Scatter Search General Scheme}
  \label{fig:SS_GS}
\end{figure}


A Scatter-Search
with a dynamic reference set
was implemented,
and a path relinking method
is used as combination method.
We used a dynamic reference set
given that the number of solutions generated
in each combination is large.
To enhanced the solutions
two improvement methods are included,
one proposed by Berman \cite{berman1987stochastic},
and a proposed Local Search.
%\todo[inline]{Agregar background de Scatter Search}

In Algorithm \ref{alg:SSInitial} (see below)
a Reference Set is created,
consisting in the bests \textit{b} solutions,
and the most diverse \text{b} solutions.
To measure the diversity of a solution,
the minimum cost perfect matching
with each of the solutions of the Reference Set.
\begin{algorithm}
  \caption{Scatter Search Initial Phase} \label{alg:SSInitial}
  \begin{algorithmic}
    \Procedure{Initial Phase}{} 
    \State $Iteration \gets 0$
    \State $IterationsUnchanged \gets 0$
    \State $RefSet \gets EmptySet()$
    \Repeat
    \State $Sols \gets DiversificationGenerator()$
    \State $Improvement(Sols)$
    \Comment {either of the two methods proposed}
    \State $RefSet.Update(Sols)$
    \State $Iteration \gets Iteration + 1$
    \State $IterationsUnchanged \gets IterationsUnchanged + 1$
    \If {$HasNewSolutions(RefSet)$}
    \State $IterationsUnchanged \gets 0$
    \EndIf
    \Until {$IterationsUnchanged \geq MaxIterUnchanged$
      \textbf{or} $Iteration \geq MaxIter$}
    \EndProcedure
  \end{algorithmic}
\end{algorithm}


In Algorithm \ref{alg:SSPhase} (see bellow),
each pair of solutions
is combined using path relinking.
However,
since the Reference Set is dynamic
some solutions are displaced
before been combined.
\todo{Aclarar donde se manda llamar algoritmo 1}
\begin{algorithm}
  \caption{General Scheme of the Scatter Search} \label{alg:SSPhase}
  \begin{algorithmic}
    \Procedure{Scatter Search Phase}{$RefSet$}
    \Repeat
    \State $G_{X} \gets SubsetGeneration(RefSet)$
    \ForAll {$X \in G_X$}
    \If {$X \subset RefSet$}
    \Comment {Some element of $X$ may have been removed}
    \State $C_{X} = SolutionCombination(X)$
    \ForAll{$Sol \in C_X$}
    \State $Improvement(Sol)$
    \EndFor
    \State $RefSet.Update(C_{X})$
    \EndIf \EndFor
    \Until {$NoNewSolutions(RefSet)$}
    \EndProcedure
  \end{algorithmic}
\end{algorithm}

