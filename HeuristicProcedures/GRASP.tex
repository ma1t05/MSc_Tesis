\subsubsection{GRASP}
%Why did you choose GRASP?
We choose GRASP
because
in the proposed problem
any allocation of sites for adjusters
is a feasible solution,
so it was decided to give some intelligence
to this simple allocation
instead of choose
random points.
%What does it consist?
We start with a partial solution
(location of a smaller number of adjusters)
with only one adjuster allocated,
evaluate each site with the greedy function,
and choose one from the best $\alpha$ evaluations,
until each adjuster was locate.
%Which evaluation functions did you use?
Several functions were evaluated
for constructive algorithm
but because presented cooperativeness
it is difficult
to approximate the final results
in a partial solution
with a greedy function.
We use three greedy functions
to test the construction.
%Fist function
The first function,
was the p-mean
or the sum the distances
of each allocation 
from each located adjuster
with their nearest demand points.
\begin{equation}
  \label{eq:grasp1}
  \sum_{j=1}^{m}{
    \sum_{i=1}^{n}{
      t_{ij}y_{ij}^{1}
    }
  }
\end{equation}
%Second function
For the second function,
we try to incorporate cooperatively,
including in the evaluation
the distance of allocations
from each located adjuster,
with their $k$ nearest demand points
plus \textit{idle} probability
given that is allocated
to their $k-1$ nearest demand points.
\begin{equation}
  \label{eq:grasp2}
  \sum_{j=1}^{m}{
    \sum_{k=1}^{p}{
      \sum_{i=1}^{n}{
        h_{ij}^{k}t_{ij}y_{ij}^{k}
      }
    }
  }
\end{equation}
%Third function
The third function,
consist in use the Mean Service Time (MST) calibration method
proposed by Jarvis \cite{jarvis1985approximating},
to obtain more accuracy values
of the current mean response time.
