%\subsubsection{GRASP}
%Why did you choose GRASP?
We chose GRASP
because
in the proposed problem
any allocation of sites for adjusters
is a feasible solution,
thus it was decided to give some intelligence
to this simple allocation
instead of choosing
random points.
%What does it consist?
In a given iteration of GRASP,
we have a partial solution (\textit{X})
where some adjusters have already been located.
The algorithm has now to decide
where to place the next adjuster.
To this end,
a greedy function is evaluated
for each potential site,
and following the GRASP philosophy,
a restricted candidate list (RCL)
is former with all the candidate sites
whose greedy function value fall
within $\alpha$- of the best possible value.
The parameter $\alpha$
is know
as the GRASP quality function threshold parameter.
Then an element form the RCL is randomly chosen
and the adjusters is located at this site.
The procedure iteratively proceeds until $p$ adjusters
have been located.
%Which evaluation functions did you use?
Several functions were evaluated
for the constructive algorithm
%but because presented cooperativeness
%it is difficult
%to approximate the final results
%in a partial solution
%with a greedy function.
We used three greedy functions
to test the construction.
%Fist function
The first function,
was the p-mean
consisting
on the sum of the distances
between each demand point 
and their nearest adjuster.
\begin{equation}
  \label{eq:grasp1}
  f_{1}(r)=
  \sum_{j\in\math{X}}{
    \sum_{i\in \math{R}_{1}(j,\math{X}\cup\{r\})}{
      t_{ij}
    }
  }
  +
  \sum_{i\in\math{R}_{1}({r},\math{X})}{
    t_{ir}
  }
\end{equation}
where
$\math{R}_{k}(j\in\math{W},\math{Y \subset \math{W}})
=\{i\in\math{V}\mid|S_{ij}\cap\math{Y}|=k-1\}$


%Second function
For the second function,
%%\todo{Refrasear y simplificar descripcion}
%we try to incorporate cooperatively,
include in the evaluation
the distances
from each demand point 
with their $\ell$ nearest adjusters,
weighted by \textit{idle} probability
given that is allocated
at order $k$.
\begin{equation}
  \label{eq:grasp2}
  f_{2}(r)=
  \sum_{k=1}^{\ell}{
    \sum_{j\in\math{X}}{
      \sum_{i\in\math{R}_{k}(j,\math{X}\cup\{r\})}{
        h_{ij}^{k}t_{ij}
      }
    }
  }
  +
  \sum_{k=1}^{\ell}{
    \sum_{i\in\math{R}_{k}(r,\math{X})}{
      h_{ir}^{k}t_{ir}
    }
  }
\end{equation}
%Third function
The third function
uses the Mean Service Time (MST) calibration method
proposed by Jarvis \cite{jarvis1985approximating},
to obtain more accuracy values
of the current mean response time.
