\section{Description of Components}
The components of Scatter Search template \cite{glover1998template}
consist of
specific subroutines of the following types:
\begin{itemize}
\item A Diversification Generator
\item An Improvement Method
\item A Reference Set Update Method
\item A Subset Generation Method
\end{itemize}
We described below
each of the components.

\subsection{A Diversification Generator}
To generate
a collection of diverse trial solutions,
using an arbitrary trial solution
(or seed solution) as an input.
\begin{algorithm}
  \caption{Path Relinking Initial Phase}\label{pr_init}
  \begin{algorithmic}[0]
    \Procedure{PathRelinking}{$Instance$}
    \State $EliteSols \gets MultiStart(Instance)$
    \Repeat
    \State $MiscSols \gets GenerateMiscSols(Instance,EliteSols)$
    \ForAll{$x \in MiscSols$}
    \State $ImprovementMethod(x)$
    \State $EliteSols.Update(x)$
    \EndFor
    \Until{$PerfectMatchingCost(MiscSols,EliteSols) < \epsilon$}
    \State $SubsetControl(EliteSols)$
    \EndProcedure
  \end{algorithmic}
\end{algorithm}

In our case,
the generator was a multi-start
consisting of a GRASP
\subsubsection{GRASP}
%Why did you choose GRASP?
We choose GRASP
because
in the proposed problem
any allocation of sites for adjusters
is a feasible solution,
so it was decided to give some intelligence
to this simple allocation
instead of choose
random points.
%What does it consist?
We start with a partial solution
(location of a smaller number of adjusters)
with only one adjuster allocated,
evaluate each site with the greedy function,
and choose one from the best $\alpha$ evaluations,
until each adjuster was locate.
%Which evaluation functions did you use?
Several functions were evaluated
for constructive algorithm
but because presented cooperativeness
it is difficult
to approximate the final results
in a partial solution
with a greedy function.
We use three greedy functions
to test the construction.
%Fist function
The first function,
was the p-mean
or the sum the distances
of each allocation 
from each located adjuster
with their nearest demand points.
\begin{equation}
  \label{eq:grasp1}
  \sum_{j=1}^{m}{
    \sum_{i=1}^{n}{
      t_{ij}y_{ij}^{1}
    }
  }
\end{equation}
%Second function
For the second function,
we try to incorporate cooperatively,
including in the evaluation
the distance of allocations
from each located adjuster,
with their $k$ nearest demand points
plus \textit{idle} probability
given that is allocated
to their $k-1$ nearest demand points.
\begin{equation}
  \label{eq:grasp2}
  \sum_{j=1}^{m}{
    \sum_{k=1}^{p}{
      \sum_{i=1}^{n}{
        h_{ij}^{k}t_{ij}y_{ij}^{k}
      }
    }
  }
\end{equation}
%Third function
The third function,
consist in use the Mean Service Time (MST) calibration method
proposed by Jarvis \cite{jarvis1985approximating},
to obtain more accuracy values
of the current mean response time.


\subsection{An Improvement Method}
To transform
a trial solution
into one or more enhanced trial solutions. 
(
If no improvement
of the input trial solution results, 
the ``enhanced'' solution
is considered to be
the same as the input solution.)

We have two improvement methods,
the first an adaptation
of the proposed method by Berman
\todo[inline]{add cite to Berman}
and the second,
a Local Search
described in the next section.

\subsection{A Reference Set Update Method}
To build and maintain a Reference Set
consisting of the \textit{b} best solutions found
(where the value of \textit{b}
is typically small,
e.g., between 20 and 40),
organized to provide efficient accessing
by other parts of the method.

We choose
a two tier Reference Set
to maintain a part of diverse solutions,
this because the combination method
generates similar solutions to input.
And we opt for be dynamic
since the number
of generated solutions
is big (proportional to \textit{p}).

\subsection{A Subset Generation Method}
To operate on the Reference Set,
to produce a subset of its solutions
as a basis for creating combined solutions.
\begin{algorithm}
  \caption{Subsets Generator}\label{ss_subsets}
  \begin{algorithmic}[0]
    \Procedure{GenerateSubsets}{$RefSet,NowTime$}
    \State $NewSols \gets SolutionsSince(RefSet,NowTime)$
    \ForAll{$(Sol_x,Sol_y) \in NewSols$}
    \If{$Sol_x \in RefSet$ \textbf{and} $Sol_y \in RefSet$}
    \State $CombinedSols \gets PathRelinkingCombination(Sol_x,Sol_y)$
    \State $Update(RefSet,CombinedSols)$
    \EndIf \EndFor
    \State $UpdateDiversity(RefSet)$
    \If{$NumberOfOldSols(RefSet,NowTime) > 0$}
    \State $OldSols \gets SolutionsUntil(RefSet,NowTime)$
    \ForAll{$Sol_x \in NewSols$}
    \ForAll{$Sol_y \in OldSols$}
    \If{$Sol_x \in RefSet$ \textbf{and} $Sol_y \in RefSet$}
    \State $CombinedSols \gets PathRelinkingCombination(Sol_x,Sol_y)$
    \State $Update(RefSet,CombinedSols)$
    \EndIf \EndFor
    \State $UpdateDiversity(RefSet)$
    \EndFor \EndIf
    \EndProcedure
  \end{algorithmic}
\end{algorithm}

We only look for subsets of size two
i.e. pairs of solutions,
because our Reference Set is dynamic,
to identify the new solutions,
we label solutions as new and old
at the start of each iteration.
We combine first the new solutions
between them,
next the new solutions with the old,
the new solutions that enters in the Reference Set
as a combined solutions,
is not part of the subsets
until the next iteration.
Each solution generated
and did not enter in the Reference Set,
displaced by a better solution,
or actually
a member of the diverse part of the Reference Set
is evaluated to be in the Reference Set
as a diverse solution.

\subsection{A Solution Combination Method}
To transform a given subset of solutions 
produced by the Subset Generation Method
into
one or more combined solution vectors.
\begin{algorithm}
  \caption{Path Relinking Combination Method}\label{alg:pr_combination}
  \begin{algorithmic}[0]
    \Procedure{PahtRelinkingCombination}{$Sol_x,Sol_y$}
    \State $CombinedSols \gets EmptyList()$
    \State $match \gets Matching(Sol_x,Sol_y)$
    \Comment{perfect,workload,random}
    \State $order \gets ProcessOrder(Sol_x,match,Sol_y)$
    \Comment{nearest/farthest first,random}
    \For{$i \gets 1,p$}
    \State $j \gets order[i]$
    \If{$Sol_x.ServerLocation(j) != Sol_y.ServerLocation(match[j])$}
    \State $Sol_x.SetServerLocation(j,Sol_y.ServerLocation(match[j]))$
    \State $CombinedSols.insert(Sol_x)$
    \EndIf \EndFor
%    \State \textbf{return} $CombinedSols$
    \EndProcedure
  \end{algorithmic}
\end{algorithm}

We choose a path relinking as a combination method
consisting of
determine a match between servers,
this match can be
\begin{itemize}
\item Perfect Matching:
  minimizing the distance between paired servers
\item Workload Matching:
  sorting the servers of both solutions
  according to the workloads,
  and match them according these sorted lists.
\item Random Matching
\end{itemize}
once we have the matching
we proceed from one solution
to interchange each server
(one by step)
until end in the other solution,
in each interchange
we have a new solution.
To make the interchanges
we have three options
\begin{itemize}
\item Nearest First
\item Farthest First
\item Random
\end{itemize}
\subsection{Initial Phase}
\begin{enumerate}
\item \textit{Seed Solution Step.}
  Create one or more seed solutions,
  which are
  arbitrary trial solutions
  used to
  initiate
  the remainder of the method.
\item \textit{Diversification Generator.}
  Use the Diversification Generator
  to generate
  diverse trial solutions
  from the seed solution(s).
\item \textit{Improvement and Reference Set Update Methods.}
  For each trial solution
  produced in Step 2,
  use the Improvement Method
  to create one or more enhanced trial solutions.
  During successive applications of this step,
  maintain and update
  a Reference Set
  consisting of the \textit{b} best solutions found.
\item \textit{Repeat.}
  Execute Steps 2 and 3
  until
  producing some designated
  total number of enhanced trial solutions
  as a source of candidates
  for the Reference Set.
\end{enumerate}

\subsection{Computational Results}
To evaluate this heuristic
several instances was tested
and
a desing of experiment was made
to determine
the match method
and the processing order method
that produce better results.

\begin{center}
  \begin{tabular}{|l|l|l|r|r|r|}
    \hline
    M & N & p & Improvement & Time (sec) & MST(sec) \\ \hline
    50 & 30 & 7 & 6.97 & 1.55 & 1.64 \\
    &  & 10 & 9.62 & 5.20 & 4.57 \\
    &  & 15 & 6.95 & 8.55 & 6.49 \\
    &  & 20 & 3.67 & 13.79 & 8.80 \\ \cline{2-6}
    & 50 & 7 & 9.76 & 2.47 & 2.41 \\
    &  & 10 & 12.20 & 7.09 & 6.15 \\
    &  & 15 & 9.75 & 11.35 & 8.51 \\
    &  & 20 & 10.63 & 30.22 & 19.38 \\ \cline{2-6}
    & 75 & 7 & 11.64 & 2.43 & 2.37 \\
    &  & 10 & 14.50 & 6.16 & 5.35 \\
    &  & 15 & 18.26 & 19.44 & 14.08 \\
    &  & 20 & 13.59 & 44.19 & 28.05 \\ \hline
    Total Result &  &  & 10.63 & 12.71 & 8.98 \\ 
    \hline
  \end{tabular}
\end{center}

\begin{center}
  \begin{tabular}{|l|l|l|r|r|r|}
    \hline
    M & N & p & Improvement & Time (sec) & MST(sec) \\ \hline
    100 & 30 & 7 & 6.70 & 4.10 & 4.33 \\
    &  & 10 & 5.74 & 9.97 & 9.65 \\
    &  & 15 & 3.38 & 14.90 & 13.46 \\
    &  & 20 & 2.80 & 28.30 & 24.01 \\ \cline{2-6}
    & 50 & 7 & 7.68 & 4.23 & 4.46 \\
    &  & 10 & 8.09 & 11.01 & 10.59 \\
    &  & 15 & 9.55 & 25.72 & 22.36 \\
    &  & 20 & 9.51 & 61.22 & 49.41 \\ \cline{2-6}
    & 75 & 7 & 7.70 & 4.55 & 4.76 \\
    &  & 10 & 11.46 & 10.17 & 9.82 \\
    &  & 15 & 11.95 & 36.70 & 31.19 \\ \hline
    Total Result &  &  & 7.69 & 19.17 & 16.73 \\ 
    \hline
  \end{tabular}
\end{center}

\missingfigure[figwidth=6cm]{Scatter Search results}

