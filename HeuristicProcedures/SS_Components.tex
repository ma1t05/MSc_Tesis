\section{Description of Components}
The components of Scatter Search template \cite{glover1998template}
consist of
specific subroutines of the following types:
\begin{itemize}
\item A Diversification Generator
\item An Improvement Method
\item A Reference Set Update Method
\item A Subset Generation Method
\end{itemize}
We described below
each of the components.

\subsection{A Diversification Generator}
To generate
a collection of diverse trial solutions,
using an arbitrary trial solution
(or seed solution) as an input.
\begin{algorithm}
  \caption{Path Relinking Initial Phase}\label{pr_init}
  \begin{algorithmic}[0]
    \Procedure{PathRelinking}{$Instance$}
    \State $EliteSols \gets MultiStart(Instance)$
    \Repeat
    \State $MiscSols \gets GenerateMiscSols(Instance,EliteSols)$
    \ForAll{$x \in MiscSols$}
    \State $ImprovementMethod(x)$
    \State $EliteSols.Update(x)$
    \EndFor
    \Until{$PerfectMatchingCost(MiscSols,EliteSols) < \epsilon$}
    \State $SubsetControl(EliteSols)$
    \EndProcedure
  \end{algorithmic}
\end{algorithm}

In our case,
the generator was a multi-start
consisting of a GRASP
\subsubsection{GRASP}
%Why did you choose GRASP?
We choose GRASP
because
in the proposed problem
any allocation of sites for adjusters
is a feasible solution,
so it was decided to give some intelligence
to this simple allocation
instead of choose
random points.
%What does it consist?
We start with a partial solution
(location of a smaller number of adjusters)
with only one adjuster allocated,
evaluate each site with the greedy function,
and choose one from the best $\alpha$ evaluations,
until each adjuster was locate.
%Which evaluation functions did you use?
Several functions were evaluated
for constructive algorithm
but because presented cooperativeness
it is difficult
to approximate the final results
in a partial solution
with a greedy function.
We use three greedy functions
to test the construction.
%Fist function
The first function,
was the p-mean
or the sum the distances
of each allocation 
from each located adjuster
with their nearest demand points.
\begin{equation}
  \label{eq:grasp1}
  \sum_{j=1}^{m}{
    \sum_{i=1}^{n}{
      t_{ij}y_{ij}^{1}
    }
  }
\end{equation}
%Second function
For the second function,
we try to incorporate cooperatively,
including in the evaluation
the distance of allocations
from each located adjuster,
with their $k$ nearest demand points
plus \textit{idle} probability
given that is allocated
to their $k-1$ nearest demand points.
\begin{equation}
  \label{eq:grasp2}
  \sum_{j=1}^{m}{
    \sum_{k=1}^{p}{
      \sum_{i=1}^{n}{
        h_{ij}^{k}t_{ij}y_{ij}^{k}
      }
    }
  }
\end{equation}
%Third function
The third function,
consist in use the Mean Service Time (MST) calibration method
proposed by Jarvis \cite{jarvis1985approximating},
to obtain more accuracy values
of the current mean response time.


\subsection{An Improvement Method}
To transform
a trial solution
into one or more enhanced trial solutions. 
(
If no improvement
of the input trial solution results, 
the ``enhanced'' solution
is considered to be
the same as the input solution.)

We have two improvement methods,
the first an adaptation
of the proposed method by Berman et al. \cite{berman1987stochastic}
and the second,
a Local Search
described in the next section.
\subsection{Proposed Local Search}
%the move
The move considered
is to relocate an agent from its current position
to a different position.
%the neighborhood
The entire neighborhood
consist of all possible moves
that can take place from a given solution.
However,
due to the high cost of move relocation assessment,
a reduced neighborhood is considered.
Thus,
instead of considering all possible moves,
the procedure focuses on relocating the adjuster
with the smallest workload
to a place around or near the adjuster with the largest workload.
\missingfigure[figwidth=6cm]{Example of less workload adjuster}
\missingfigure[figwidth=6cm]{Example of neighborhood to evaluate}


\subsection{A Reference Set Update Method}
To build and maintain a Reference Set
consisting of the \textit{b} best solutions found
(where the value of \textit{b}
is typically small,
e.g., between 20 and 40),
organized to provide efficient accessing
by other parts of the method.

We choose
a two tier Reference Set
to maintain a part of diverse solutions,
this because the combination method
generates similar solutions to input.
And we opt for be dynamic
since the number
of generated solutions
is big (proportional to \textit{p}).

Whenever the Reference Set is update
it is try to incorporate
the solutions generated in a combination,
because the Reference Set is divided
in two parts,
first try to accommodate each solution
by quality criteria.
If a solutions enters the Reference Set,
degrades solutions lower quality to it,
removing the worst solution of the Reference Set.

After trying to accommodate solutions
for quality criterion,
solutions that do not enter the Reference Set
and those that are removed,
they are try to enter with diversity criterion.
reassessing the diverse solutions,
since they evaluation depends
on the quality members.

\subsection{A Subset Generation Method}
To operate on the Reference Set,
to produce a subset of its solutions
as a basis for creating combined solutions.
\begin{algorithm}
  \caption{Subsets Generator}\label{ss_subsets}
  \begin{algorithmic}[0]
    \Procedure{GenerateSubsets}{$RefSet,NowTime$}
    \State $NewSols \gets SolutionsSince(RefSet,NowTime)$
    \ForAll{$(Sol_x,Sol_y) \in NewSols$}
    \If{$Sol_x \in RefSet$ \textbf{and} $Sol_y \in RefSet$}
    \State $CombinedSols \gets PathRelinkingCombination(Sol_x,Sol_y)$
    \State $Update(RefSet,CombinedSols)$
    \EndIf \EndFor
    \State $UpdateDiversity(RefSet)$
    \If{$NumberOfOldSols(RefSet,NowTime) > 0$}
    \State $OldSols \gets SolutionsUntil(RefSet,NowTime)$
    \ForAll{$Sol_x \in NewSols$}
    \ForAll{$Sol_y \in OldSols$}
    \If{$Sol_x \in RefSet$ \textbf{and} $Sol_y \in RefSet$}
    \State $CombinedSols \gets PathRelinkingCombination(Sol_x,Sol_y)$
    \State $Update(RefSet,CombinedSols)$
    \EndIf \EndFor
    \State $UpdateDiversity(RefSet)$
    \EndFor \EndIf
    \EndProcedure
  \end{algorithmic}
\end{algorithm}

We only look for subsets of size two
i.e. pairs of solutions,
because our Reference Set is dynamic,
to identify the new solutions,
we label solutions as new and old
at the start of each iteration.
We combine first the new solutions
between them,
next the new solutions with the old,
the new solutions that enters in the Reference Set
as a combined solutions,
is not part of the subsets
until the next iteration.
Each solution generated
and did not enter in the Reference Set,
displaced by a better solution,
or actually
a member of the diverse part of the Reference Set
is evaluated to be in the Reference Set
as a diverse solution.

\subsection{A Solution Combination Method}
To transform a given subset of solutions 
produced by the Subset Generation Method
into
one or more combined solution vectors.

We choose a path relinking as a combination method
consisting of
determine a match between servers,
this match can be
\begin{itemize}
\item Perfect Matching:
  minimizing the distance between paired servers
\item Workload Matching:
  sorting the servers of both solutions
  according to the workloads,
  and match them according these sorted lists.
\item Random Matching
\end{itemize}
once we have the matching
we proceed from one solution
to interchange each server
(one by step)
until end in the other solution,
in each interchange
we have a new solution.
To make the interchanges
we have three options
\begin{itemize}
\item Nearest First
\item Farthest First
\item Random
\end{itemize}
\begin{algorithm}
  \caption{Path Relinking Combination Method}\label{alg:pr_combination}
  \begin{algorithmic}[0]
    \Procedure{PahtRelinkingCombination}{$Sol_x,Sol_y$}
    \State $CombinedSols \gets EmptyList()$
    \State $match \gets Matching(Sol_x,Sol_y)$
    \Comment{perfect,workload,random}
    \State $order \gets ProcessOrder(Sol_x,match,Sol_y)$
    \Comment{nearest/farthest first,random}
    \For{$i \gets 1,p$}
    \State $j \gets order[i]$
    \If{$Sol_x.ServerLocation(j) != Sol_y.ServerLocation(match[j])$}
    \State $Sol_x.SetServerLocation(j,Sol_y.ServerLocation(match[j]))$
    \State $CombinedSols.insert(Sol_x)$
    \EndIf \EndFor
%    \State \textbf{return} $CombinedSols$
    \EndProcedure
  \end{algorithmic}
\end{algorithm}

