% Validating and applying a model 
% for locating emergency medical vehicles in Tucson, AZ
% Goldberg 1990

\section{Model A}

This model is based
on the model
proposed by Goldberg \cite{goldberg1990validating},
and it contains assignment variables
for all possible orders.

The following assumptions are considered in the model:
\begin{itemize}
\item The probability that an adjuster is busy
  is $\rho$ and is unaffected by the state of the system.
\item There is a strict ordering of the basis preferred for each zone
  that does not depend
  on the current state of the system.
\item All calls are answered
  by an adjuster originating from its base,
  not in route back to the base.
\item The arrival of calls to the system
  follows a stationary distribution.
\item The model is presented
  using a 0-queue assumption.
\end{itemize}

Sets and indexes:
\begin{itemize}
\item \textit{n} number of demand points
\item \textit{m} number of potential sites to locate adjusters/facilities
\item \textit{p} number of available adjusters
\item \textit{i} index for demand points;
  $i \in V = \{1,2,\ldots,n\}$ 
\item \textit{j} index for potential site for adjusters/facilities
  $j \in W = \{1,2,\ldots,m\}$
\item $k$ index for possible order;
  $k \in K = \{1,2\ldots,p\}$
\item $S_{ij} = \{
  r \in W | \mbox{ site } r \mbox{ is preferred by proximity before site } j
  \mbox{ for demand point } i
  \}$
\end{itemize}

Parameters:
\begin{itemize}
\item $\lambda_i$ arrival rate of calls
  for demand point \textit{i}
\item $\rho$ is the utilization of each adjuster,
  the value is between 0 and 1, 
  where 0 means that the server is always idle.
  To obtain an approximate value for $\rho$
  we use the formula proposed by Berman et al.
  \cite{berman1982median}
  \begin{*equation}
    \rho = \frac{\sum_{i=0}^{n}{\lambda_i}}{m p}
  \end{*equation}
\item $t_{ij}$ is the expected travel time
  between demand point \textit{i}
  and potential site \textit{j}.
\item $h_{ij}^{k}$ is the probability
  that $j$ serves point $i$
  given that
  it is the $k$-th preferred.
  It is calculated
  using the following formula:
  \begin{*equation}
    {h}_{ij}^{k} = (1-\rho)\rho^{k-1}
  \end{*equation}
\end{itemize}

Variables:
\begin{itemize}
\item $x_j =
  \begin{cases} 
    1 & \mbox{if an adjuster is placed at potential site } j \\
    0 & \mbox{otherwise}
  \end{cases}$
\item $y_{ij}^{k} =
  \begin{cases} 
    1 & \mbox{if adjuster in } j \mbox{, is the }
    k\mbox{-th to cover demand point } i \\
    0 & \mbox{otherwise}
  \end{cases}$
\end{itemize}

Model
{\small
  \begin{equation}
    \min \, \sum_{j=1}^{m}{
      \sum_{k=1}^{p}{
        \sum_{i=1}^{n}{
          h_{ij}^{k}t_{ij}y_{ij}^{k}
        }
      }
    }
  \end{equation}
}
Minimize the average expected response time
subject to
\begin{align}
  \label{eq:2}
  \sum_{j \in W}{x_j}
  & = p
\end{align}
Only locate $p$ adjusters
\begin{align}
  \label{eq:3}
  \sum_{j \in W}{y_{ij}^{k}}
  & = 1
  & i \in V, k
  &\in K
\end{align}
Each demand point $i$ is covered by an adjuster on each order $k$
\begin{align}
  \label{eq:4}
  y_{ij}^{k}
  & \leq x_j
  & i \in V,j \in W, k
  &\in K
\end{align}
Relationship between variables \textit{x} and \textit{y}
\begin{align}
  \label{eq:5}
  \sum_{k = 1}^{p}{
    y_{ij}^{k}
  }
  & \leq x_j
  & i \in V, j 
  & \in W
\end{align}
For each located adjuster,
there can only be
a maximum of one ordered assignment.
\begin{align}
  y_{ij}^{k} 
  & \leq \sum_{r\in S_{ij}}{y_{ir}^{k-1}}
  & i \in V,j \in W, k
  & \in K\setminus\{1\}
\end{align}
Assign \textit{j} to cover \textit{i} in order $k$
only if
the assignment of order $k-1$
was made for some $r \in S_{ij}$
\begin{align}
  x_{j}
  & \in \{0,1\}
  & j 
  & \in W \nonumber
  \\
  y_{ij}^{k}
  & \in \{0,1\}
  &  i \in V,j \in W,k
  &\in K \nonumber
\end{align}

% Model implications
Observe that
we do not need
to add a constraint
to ensure
the counterpart of (\ref{eq:5})
because (\ref{eq:2}) and (\ref{eq:4})
ensure that
each adjuster
must cover each demand point
for some order,
therefore
if an adjuster located at \textit{j}
does not cover demand point \textit{i}
at order \textit{k}
(indicated by
the maximum covering order in $S_{ij}$)
there will be at least
one adjuster
that does not cover
demand point \textit{i}
at any order
resulting in an infeasible solution.
