\section{Comparison}
As you can see,
the model A
has a considerable amount of binary variables,
it is why the model B was formulated
but due the lack of allocation variables
we need to add other variables and constraints
to ensure a similar behavior.

We can determine the size of the models
in function of \textit{N}, \textit{M}, \textit{p} and $\ell$.

\begin{table}[h]
  \centering
  \begin{tabular}{c|c|c|}
    \cline{2-3}
    & Model A & Model B \\ \hline
    \multicolumn{1}{|l|}{variables} &
    $m(np+1)$ &
    $m(n(\ell+2)+1)$ \\ \hline
    \multicolumn{1}{|l|}{constraints} &
    $n(2mp+p)+1$ &
    $n((l+8)m+1)+1$ \\ \hline
  \end{tabular}
  \caption{Models size}
\end{table}

And we can notice
that since $\ell < p - 2$
have fewer variables
in Model B,
and when $\ell < 2p - 8$
we obtain less constrains
than in Model A.
