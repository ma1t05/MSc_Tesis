\section{Comparison}
As can be seen,
model A
has a considerable amount of binary variables,
therefore model B was formulated.
However, due the lack of allocation variables
we needed to add more variables and constraints
to ensure a similar behavior.

The size of the models
as a function of \textit{n}, \textit{m}, \textit{p} and $\ell$
is shown in table \ref{tab:modsize}

\begin{table}[h]
  \label{tab:modsize}
  \centering
  \begin{tabular}{c|c|c|}
    \cline{2-3}
    & Model A & Model B \\ \hline
    \multicolumn{1}{|l|}{variables} &
    $m(np+1)$ &
    $m(n(\ell+2)+1)$ \\ \hline
    \multicolumn{1}{|l|}{constraints} &
    $n(2mp+p)+1$ &
    $n((l+8)m+1)+1$ \\ \hline
  \end{tabular}
  \caption{Models size}
\end{table}

Since $\ell < p - 2$
one can notice
that model B
has lees variables
than model A,
and when $\ell < 2p - 8$
model B has
less constraints too.
