\section{Comparison betweem models}
As can be seen,
model A
has a considerable amount of binary variables.
This is why model B was introduced;
however, due the lack of allocation variables
we needed to add more variables and constraints
to ensure a similar behavior.

The size of models
as function of \textit{n}, \textit{m}, \textit{p} and $\ell$
as shown in Table \ref{tab:modsize}

\begin{table}[h]
  \label{tab:modsize}
  \centering
  \begin{tabular}{c|c|c|}
    \cline{2-3}
    & Model A & Model B \\ \hline
    \multicolumn{1}{|l|}{variables} &
    $m(np+1)$ &
    $m(n(\ell+2)+1)$ \\ \hline
    \multicolumn{1}{|l|}{constraints} &
    $n(2mp+p)+1$ &
    $n((l+8)m+1)+1$ \\ \hline
  \end{tabular}
  \caption{Models size}
\end{table}

We can notice that
since $\ell < p - 2$
model B has fewer binary variables.
When $\ell < 2p - 8$
model B has
less constraints than Model A.
