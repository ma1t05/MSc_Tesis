\section{Problem Statement}

% Ideal Situation
The location and dispatching policies
used by insurance agencies
should aim
to arrive to the accident areas
as early as possible,
due to several reasons
such as:
\begin{enumerate}
\item providing a timely service to their customers
\item helping clear out the accident area
\item keeping the workloads of its adjusters
  as balanced as possible
\end{enumerate}
The location policies
should be optimized for service time
but they should also consider
cooperation based on adjuster workload.
% Problem
However,
current insurance agency policies
are empirical,
\todo[inline]{Explicar por que las politicas son empiricas}
and do not consider cooperation.
By neglecting this,
adjusters tend to take
more time to arrive at the accident site,
making the insurance agency
less competitive,
and generating more traffic congestion.

% Financial Costs
The financial costs
of applying
an empirical policies
instead of optimum policies
is difficult to measure.
However the costs include
more use of fuel,
more use of vehicles
that implies
more maintenance costs,
and the opportunity cost
of losing a customer
for low quality-of-service.
% Evidence
The use of quantitative models
may also help
in what-if analysis
to assess the overall service rate
if more adjusters are placed.
% Solution
The use of mathematical models
to determine better location policies
based on scenarios,
and the use of real data
to simulate and evaluate new scenarios
versus the current policy
is one of the main contribution
on this work.

%  Benefits
The benefit of this framework
is that several policies
can be evaluated for different scenarios
(for instance high level of congestion in rainy days),
and the best policy for each scenario can be determined.

% Summary
In summary
the location of adjusters
could be improved
with the use of
mathematical models and simulation,
and obtain several benefits.

% Thesis statement
The problem studied in this thesis
consist of
given a number of adjusters,
a set of potential site for place them (\textbf{basis}),
and a set of demand points,
we have to determine
where to place the adjusters,
so as to minimize 
the average response time,
assuming
that calls
arrive with a Poisson distribution
and with an own arrival rate
for each demand point.
% Conceptual Problems
