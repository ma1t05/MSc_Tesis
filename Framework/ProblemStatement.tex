\chapter{Problem Statement}
In this chapter we describe the problem
mentioned previously

% Ideal Situation
The location and dispatching policies
used by insurance agencies
should aim
to arrive to incidents
early as possible,
so that
the workloads for adjusters are balanced.
The location policies
should be optimized for service time
but also should consider
cooperatively
based on workloads.

% Problem
However,
current insurance agency policies
are empirical,
and does not consider cooperative.
Taking more time to arrive at incidents,
making the insurance agency
less competitive,
and generating more traffic congestion.

% Financial Costs
The financial costs
of apply
empirical policies for location
instead of optimum policies
are difficult to measure,
but the costs include
more use of fuel,
more use of vehicles
that implies
more maintenance costs,
and less customers
that prefer a faster service.

% Evidence
Currently
there are other
insurance companies
which have
better response times,
and
it is know
that the amount of adjusters
is similar,
so,
there must be
a way to improve the service
at least at their same level.

% Solution
We propose
the use of mathematical models
to determine better policies of location
based on scenarios
of real data
and the use simulation
to evaluate that policies
versus the actually used.

%  Benefits
The benefits of this
is that several policies
can be evaluated
for different scenarios
and determine
at planing level
the best or a better policy
for each case.

% Summary
As summary
the location of adjusters
could be improved
with the use of
mathematical models and simulation,
and obtain several benefits.

% Thesis statement

% Conceptual Problems
