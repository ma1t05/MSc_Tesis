\section{Path Relinking}
As we mentioned before,
the path relinking algorithm
consists of two key components
the match method and the processing method.

This experiment aims to determine which methods are best suited
and in order to do that,
we measure the relative improvements
achieved regarding the best solution
reported by the multi-start method.

In the Table \ref{tab:pr_imp}
the average improvement
respect the best multi-start solution is reported.
Each column
represents the results
of 180 instances
of size \texit{M} and \textit{N}
with four differents values of \textit{p},
the arrival rate ($\lambda$)
and the service rate ($\mu$)
remain constant.
\input{Experiments/PathRelinking_tabImp}
%\todo{Analizar los resultados}
The combination that offers better results
is the random matching
plus
the order of the neares first.
We can assume that is
because greater diversity
is generated by having a random matching
and also that,
from quality solutions
and apply few drastic changes
it is possible to obtain a solution with diversity and quality.
