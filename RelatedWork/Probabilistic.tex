\section{Probabilistic location models for planning ESSs}
% source: galvao2008emergency
Probabilistic location models
are optimization models
that permit a more accurate planning
of EMSs at the strategic level.
They were initially developed
based on simplifying assumptions of independence
and non-cooperation among servers,
which are seldom encountered in practice.
% eos
% source: galvao2008emergency
The location of facilities (servers) in ESSs
has as its main objective
the provision of coverage to demand areas.
The definition of
probabilistic location models for planning these systems
was a natural extension
of their deterministic equivalents,
the location models with covering constraints.
The notion of coverage
implies
the definition of a service distance (time),
which is the critical distance (time)
beyond which a demand area is considered not covered.
A demand area is
therefore
considered covered
if
it is within a predefined critical distance (say D)
from at least one of the existing facilities.
%The simplest deterministic covering model
%corresponds to the Location Set Covering Problem (LSCP),
%which provides coverage for every demand area under consideration.
%The provision of total coverage, however,
%may prove to be economically infeasible,
%in the sense that the number of servers required
%may not be compatible with the resources available to the decision-maker.
%The Maximal Covering Location Problem (MCLP)
%was defined by Church and ReVelle (1974) within this context.
%In this case the objective is to locate a number of facilities
%(say p facilities) that is compatible with the resources available,
%such that the maximal possible population of a given geographical region
% is covered within D.

A limitation of the deterministic models
is that
they assume
that servers are available when requested,
which is not always true in practical situations.
Congestion in emergency services,
which may cause
the unavailability
of servers located
within the critical distance
when a call is placed,
lead to
the development of
a second generation of location covering models
focused on additional coverage.
These models
emphasize the importance of additional coverage's
for the demand areas,
given the possibility that
in congested systems
the first server,
possibly
the only server in a particular coverage area,
might not be available when requested.
Several such models were developed,
as for example in Daskin and Stern (1981),
Eaton et al. (1981),
Hogan and ReVelle (1986)
and Batta and Mannur (1990).
Probabilistic covering models
are a natural extension
of the second-generation models.
The initial problems
defined within this context
were
the Maximum Expected Covering Location Problem (MEXCLP)
and
the Maximum Availability Location Problem (MALP).
%Extensions of these problems
%lead to the use
%of the hypercube queuing model
%in corresponding solution methodologies.
% eos
