\section{Queueing Models}
The hypercube model (denoted Hypercube)
and the hypercube approximation (denoted A-Hypercube).
developed by Larson  \cite{larson1974hypercube,larson1975approximating}
are the most well known queueing approaches.
%and extended by others (see Swersey \cite{swersey1994deployment})
% source: galvao2008emergency
These are not an optimization models;
they are only a descriptive models
that permits the analysis of scenarios.
Both models
estimate system operating characteristics
that are used to evaluate a series of objectives.
They can evaluate cooperation between vehicles,
their weaknesses, include
\begin{itemize}
\item Assumptions of an exponentially distributed service time.
\item Computational difficulties for problems with many vehicles.
\item Require that service time be solely vehicle-dependent
  rather than call location-dependent.
\end{itemize}

The computational problems are remediated in the A-Hypercube
approximating the vehicle busy probabilities
by solving a system of nonlinear equations
whose size depends on the number of vehicles.

Optimization models for locating Emergency Medical Services (EMS) bases
that use Hypercube or A-Hypercube as a function evaluation subroutine
include Jarvis' location-allocation problem \cite{jarvis1975optimization},
Berman and Larson's congested median problem \cite{berman1982median},
Benviniste's location-allocation problem \cite{benveniste1985solving}.
and Berman, Larson and Parkan's
stochastic queue p-median problem \cite{berman1987stochastic}.
These methods are heuristic local improvement approaches
that assume it is possible to locate a vehicle in every zone.

\subsection{Mean Service Calibration}
% source: goldber1990simulation
Call location-dependent service time
can be modeled using the Mean Service Calibration method (denoted MSC).
As in cases of
Jarvis \cite{jarvis1975optimization} 
and Halpern \cite{halpern1977accuracy}
where mean service time,
as opposed to the distribution of service time,
and show that
is sufficient to obtain accurate estimates of system performance.
The major shortcoming of MSC is that either
Hypercube or A-Hypercube
is evaluated in each iteration;
it can thus be a computationally expensive approach.

To eliminate the computational inefficiency of the MSC method,
Jarvis \cite{jarvis1985approximating} developed
an approximation model
for spatially distributed queueing systems
\footnote{See Appendix \ref{ch:Jarvis} for more infomation}.
The model assumes that call service time
is call location-dependent,
wherein all vehicles
have the same service rate and utilization
while service is exponentially distributed.

% eof
\subsection{Calibration Process}
% source: galvao2008emergency
% Calibration process of the mean service times
In certain EMSs
and other emergency systems,
travel times may represent
a considerable part of service times.
In such cases,
it may be advisable
to adjust the service times
by means of a calibration process,
which can be performed
using a simple iterative procedure
that is proposed by Berman et al. \cite{berman1987stochastic}.
Basically,
the procedure consists of
verifying
if there are
significant differences among the input mean service times
and the output mean service times (computed by the hypercube model).
In this case,
the hypercube is solved
using the computed mean service times as inputs,
until
the differences
among input and output values
are sufficiently small.
This procedure is called
a calibration process.
Note that
it takes into account
that the mean travel time
depends on the location of the user
and the identity of the server.
Empirical experiments show
that this procedure
usually converges in two or three iterations,
for a reasonably accurate estimation of the mean service times,
although a formal proof of the convergence of the method
is apparently not available in the literature.
