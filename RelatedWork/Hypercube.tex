\section{Hypercube}
% source: galvao2008emergency
The hypercube model
is not an optimization model; 
it is only a descriptive model
that permits the analysis of scenarios

Given a system configuration,
the hypercube model
is able to evaluate
a variety of performance measures relevant for decision-making,
either region-wide or for each server or region. 
These include
server workloads,
mean user response times,
fraction of dispatches of each server to each region,
among others.

\subsection{Calibration Process}
% source: galvao2008emergency
% Calibration process of the mean service times
In certain EMSs
and other emergency systems,
travel times may represent
a considerable part of service times.
In such cases,
it may be advisable
to adjust the service times
by means of a calibration process,
which can be performed
using a simple iterative procedure.
Basically,
the procedure consists of
verifying
if there are
significant differences among the input mean service times
and the output mean service times (computed by the hypercube model).
In this case,
the hypercube is solved
using the computed mean service times as inputs,
until
the differences
among input and output values
are sufficiently small.
This procedure is called
a calibration process.
Note that
it takes into account
that the mean travel time
depends on the location of the user
and the identity of the server.
Empirical experiments show
that this procedure
usually converges in two or three iterations,
for a reasonably accurate estimation of the mean service times,
although a formal proof of the convergence of the method
is apparently not available in the literature.
