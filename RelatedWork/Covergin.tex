\subsection{Covering models}
% source: brotcorner2003ambulance
In the location set covering model (LSCM)
introduced Toregas et al. \cite{toregas1971location}
the aim
is to minimize
the number of ambulances needed
to cover all demand points.
This model
ignores several aspects of real-life problems,
the most important
probably being that
once an ambulance is dispatched,
some demand points
are no longer covered.
However,
the authors provide
a lower bound
on the number of ambulances
required to ensure full coverage.

The maximal covering location problem (MCLP)
originally proposed by Church and ReVelle \cite{church1974maximal}
is an alternative approach
proposed to overcome some of the shortcomings of the LSCM.
In the MCLP the objective is to maximize population coverage
subject to limited ambulance availability.
% eos

A limitation of the deterministic models
is that
they assume
that servers are available when requested,
which is not always true in practical situations.
Congestion in emergency services,
which may cause
the unavailability of servers located
within the critical distance
when a call is placed,
leads to
the development of
a second generation of location covering models
focused on additional coverage.

The definition of
probabilistic location models for planning these systems
is a natural extension
of their deterministic equivalents,
the location models with covering constraints.
The notion of coverage
implies
the definition of a service distance (time),
which is the critical distance (time)
beyond which a demand area is considered not covered.
A demand area is
therefore
considered covered
if
it is within a predefined critical distance (say D)
from at least one of the existing facilities.

The Maximum Expected Covering Location Problem (MEXCLP)
defined by Daskin \cite{daskin1983maximum}
whose objective is to
maximize the expected coverage
of all demand areas under consideration,
assume that servers operate independently
and that all servers have the same busy probability (workload) $\rho$,
allowings that more than one server be situated in any given location.
% source: golberg1990validating
Daskin et al. \cite{daskin1981hierarchical},
assume
that travel times are deterministic
and coverage is an ``all-or-nothing'' property.
% eos

ReVelle et al. \cite{revelle1989maximum}
proposes two variations for the Maximum Availability Location Problem (MALP)
locate \textit{p} servers
in suchs a way
as to minimize the population
which will find a server available within $\alpha$ reliability
the firstone assume,
like Daskin,
that each server has the same busy probability,
and predetermine the number of times a demand point needs to be covered.
The other,
allow busy fractions
to be different in the various sections of a region under consideration
(but not for each server to be located)

These models
emphasize the importance of additional coverage
for the demand areas,
given the possibility that
in congested systems
the first server,
possibly
the only server in a particular coverage area,
might not be available when requested.
Gendreau et al. \cite{gendreau1997solving}
proposes a model with double coverage,
uses two radious $r_1$, and $r_2$ $(r_2 > r_1)$,
to locate $p$ ambulances,
such that
all the demand must be covered by an ambulance
located within $r_2$ time units,
and,
a proportion $\alpha$ of the demand
must also be within $r_1$ units of an ambulance,
which may or may not be the same ambulance that
covers this customer within $r_2$ time units.
Note that a feasible solution may not exist
if the parameters $r_1$, $r_2$ and $\alpha$ are too restrictive.
