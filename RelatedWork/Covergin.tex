\subsection{Covering models}
% source: brotcorner2003ambulance
In the location set covering model (LSCM)
introduced Toregas et al. \cite{toregas1971location}
the aim
is to minimize
the number of ambulances needed
to cover all demand points.
This model
ignores several aspects of real-life problems,
the most important
probably being that
once an ambulance is dispatched,
some demand points
are no longer covered.
However,
the authors provide
a lower bound
on the number of ambulances
required to ensure full coverage.

The maximal covering location problem (MCLP)
originally proposed by Church and ReVelle \cite{church1974maximal}
is an alternative approach
proposed to overcome some of the shortcomings of the LSCM.
In the MCLP the objective is to maximize population coverage
subject to limited ambulance availability.
% eos
% source: golberg1990validating
Daskin et al. \cite{daskin1981hierarchical},
and all other set covering approaches
\cite{revelle1989maximum,gendreau1997solving,saydam2003accurate},
assume that
travel times are deterministic
and coverage is an ``all-or-nothing'' property.
% eos
