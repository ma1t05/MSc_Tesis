\chapter{Methodology}
In this chapter
we describe the solution methodology proposed
to solve the described problem.

A first model was created
base on proposed by Goldberg \cite{goldberg1990validating}
for which,
we made some relaxations
instead obtaining a linear model.

Proposed a second model
that performs an additional simplification,
considering that
it is unlikely
assign an adjuster
to a demand point
covered previously
by more adjusters,
omitting allocation variables
and adding restrictions
to guarantee
the correct order of allocation.

Because the problem stochasticity
can not evaluate solutions accurately,
so the Hypercube model is used.

Several functions were evaluated
for constructive algorithm
but because presented cooperativeness
it is difficult
to approximate the final results
in a partial solution
with a greedy function.

To improve these solutions,
a scatter-search was implemented
with a dynamic reference set.
We opted for a dynamic reference set
due the number of solutions generated
in each combination is big.
