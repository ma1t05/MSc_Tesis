\chapter{Methodology}
In this chapter
we describe the solution metodology proposed
to solve the described problem.

Se creo un primer modelo
basado en lo que Goldberg propone,
para lo cual se realizaron
varias relajaciones
a diferencia de Goldber,
el modelo propuesto es lineal.

Se propuso un segundo modelo
que realiza una simplificacion mas,
tomando en cuenta que
las probabilidades de asignar
un ajustador 
a un punto de demanda
que es cubierto prebiamente por mas ajustadores
es muy poca,
se omiten variables de asignacion,
y se agregan variables y restricciones
para garantizar
el correcto orden de asignacion

Debido a la stocacidad del problema
no se puede evaluar de manera exacta
cada una de las soluciones,
para ello se utiliza el modelo del hypercubo

Se evaluaron varias funciones
para el metodo constructivo
debido a la cooperatividad que se presenta
es dificil
captar parte de los resultados finales
en una solucion parcial
con una funcion greedy.
Por lo cual se opto
por un algoritmo multi-arranque.

Para mejorar estas soluciones,
se implemento una busqueda dispersa
con dos niveles y conjunto de refencia dinamico

Se opto por mantener dos nivles
ya que
la diversidad
se perdia facilmente
al realizar las primeras cruzas,
y se opto
por el conjunto de refencia dinamico
ya que una gran cantidad de soluciones se generan
en cada cruza
