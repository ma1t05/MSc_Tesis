%%%%%%%%%%%%%%%%%%%%%
% Documento maestro %
%%%%%%%%%%%%%%%%%%%%%
\documentclass{fime}

%%%%%%%%%%%%%%%%%%%%%%%%%%%%%%%%%%%%%%%%%%%
% Cargando paquetes y definiendo opciones %
%%%%%%%%%%%%%%%%%%%%%%%%%%%%%%%%%%%%%%%%%%%
% Aquí puedes cargar los paquetes que vas a usar. La clase
% fime ya incluye babel, inputenc, graphicx y los de la AMS.
% Cargar un paquete está a tu libertad (y responsabilidad).
\usepackage{hyperref}
    \hypersetup{breaklinks=true,colorlinks=true,
        linkcolor=black,citecolor=black,urlcolor=black}
\usepackage{datetime} % monthname
\usepackage{todonotes} % temporaly use for todo notes
\usepackage{algorithm}
\usepackage{algpseudocode}

%%%%%%%%%%%%%%%%%%%%%
% Definiendo campos %
%%%%%%%%%%%%%%%%%%%%%
\def\titulo{Optimal Location of Car Wreck Adjusters}
\def\autor{Luis Alberto Maltos Ortega}
\def\matricula{1390200}
\def\grado{Maestría en Ciencias en Ingenie\'ia de Sistemas}
% En caso de que el grado tenga orientación o especialidad
% llenar el siguiente campo dejando un ESPACIO INICIAL,
% en caso contrario, dejar vacío
\def\orientacion{ }
\def\fecha{\monthname,  \the\year} % Coloca el mes con mayúscula inicial

\def\asesor{Dr. Roger Z. R\'ios Mercado}
\def\revisorA{Mar\'ia Ang\'elica Salazar Aguilar}
\def\revisorB{Mar\'ia Guadalupe Villareal Marroqu\'in}
% En el caso de que tu tesis sea de doctorado
% activa la variable cambiándola a \doctoradotrue
% y define tus otros dos revisores
\newif\ifdoctorado\doctoradofalse
\def\revisorC{Nombre del revisor C}
\def\revisorD{Nombre del revisor D}
% El visto bueno siempre va
\def\vobo{Dr. Simón Martínez Martínez}

\includeonly{
  Misc/Cover,
  Misc/Dedicatory,
  Misc/Acknowledgements,
  Misc/Abstract,
  Introduction/Main,
  RelatedWork/Main,
  Framework/Main,
  HeuristicProcedures/Main,
  Evaluation/Main,
  Conclusions/Main,
  Appendices/Jarvis,
  Appendices/Berman
}
%%%%%%%%%%%%%%%%%%%%%%%
% Inicia el documento %
%%%%%%%%%%%%%%%%%%%%%%%
\begin{document}

\frontmatter
\pagestyle{main}

%%% Incluye PortillasM si tu tesis es de Maestría
%%% y PortillasD si es de doctorado.
\include{Misc/Cover}
% Dedicatoria

\thispagestyle{empty}
\vspace*{17mm}

\begin{flushright}
\begin{itshape}

To my wife Alejandra\\
to my children
they are the reason to beat.

\end{itshape}
\end{flushright}



\tableofcontents
\listoffigures
\listoftables
\listoftodos
\todo{Remove list of to do's when complete}

%Acknowledgements

\chapter{Acknowledgments}
\markboth{Acknowledgments}{}

%Aquí puedes poner tus agradecimientos.
%(No olvides agradecer a tu comité de tesis,
I want to especially thank
Prof. Roger Z. Rios
For giving me the opportunity to work with him
and be my advisor,
for guiding me in my academic development
and for the effort and support he gave me
I also would like to thanks him
for his patience,
without his support
this work would not have been possible. 

Thanks to my thesis committee
Prof. Ma. Angelica Salazar Aguilar
and Prof. Ma. Guadalupe Villareal Marroquin,
for their comments and suggestions
that made throughou his work,
and for help me
to complete this work,

%a tus profesores,
to all professors of the PISIS
for their courses so complete and enriching,

%a la facultad
to the FIME
for the facilities granted
and ease the paperwork,

%y a CONACyT
%en caso de que hallas sido beneficiado con una beca).
to Consejo Nacional de Cíencia y Tecnología (CONACyT)
for the Graduate Fellowship
and the research project grant 2011-1-166397
with which it was possible the realization of my studies
and attendance at national conferences
to present part of this work.

%Resumen

\chapter{Abstract}
\markboth{Abstract}{}

{\setlength{\leftskip}{10mm}
\setlength{\parindent}{-10mm}

\autor.

Candidato para obtener el grado de \grado\orientacion.

\uanl.

\fime.

Título del estudio: \textsc{\titulo}.

\noindent Número de páginas: \pageref*{lastpage}.}

%%% Comienza a llenar aquí
\paragraph{Problem Description:}
When a traffic accident occurs
in cities with a large traffic flow
the roads surrounding the crash site
are affected by traffic congestions.
In some countries, such as Mexico,
even small accidents are troublesome
due to the fact
that a claim adjuster
from the car insurance company
must arrive to the site
and document the accident
before the vehicle may be moved
as required by law.
Thus,
in this particular setting,
the adjusters location
becomes a key factor in providing timely service.

\paragraph{Objectives and method of study:}
%Aquí debes poner tus objetivos y métodos de estudio.
The aim of this thesis is to
\begin{itemize}
\item Provide quantitative tools
with scientific support
for the optimal location
of car wreck adjusters.
\item Develop
adequate mathematical models
for representing
some of the important company concerns.
Design and develop efficient solution techniques
for handling real-world instances of the problem.
\item Assess
the performance of the proposed techniques
based on an appropriate experimental design.
\end{itemize}

\paragraph{Contributions and Conclusions:}
%Y aquí tus contribuciones y conclusiones.
%(También es parte del formato).
The most important contributions of this work
are the following:
\begin{itemize}
\item Two integer programming models
  are introduced
  with the aim of
  minimizing average response time.
\item The second model
  takes into account
  special features
  to reduce the number of variables.
\item A scatter search heuristic
  was designed, implemented, and tested
  on a wide set of instances
  with very good results.
\end{itemize}

\bigskip\noindent\begin{tabular}{lc}
\vspace*{-2mm}\hspace*{-2mm}Firma del asesor: & \\
\cline{2-2} & \hspace*{1em}\asesor\hspace*{1em}
\end{tabular}




\mainmatter
\pagestyle{fime}

%%% Haz un documento para cada capítulo
\chapter{Introduction}

The main idea behind this thesis
is to develop mathematical models
to improve the service offered by car insurance agents,
determining the location
of adjusters available
to improve service
helping them arrive to accident sites sooner.
And determine
the number of adjusters required
to perform the service,
within desired quality of service.

\section{Problem Statement}
A car insurance company,
which has a group of operations in Monterrey N.L.,
face with the constant concern
to locate their insurance agents (adjusters),
against various scenarios presented daily,
since its goal,
is to reach the places where an incident occurs,
as soon as possible.

This is a problem of an emergency system, % add cite
in which
it should cover the entire territory of study
with a fixed number of adjusters,
which they must be located
to reduce the average time of arrival to incidents.

\section{Background}
This is a practical problem
that has not been studied before
to the best of our knowledge.
This stems
from the fact
that in other countries,
when accident occurs
cars owners
are allowed to move their cars
from the accident
if this obstructs traffic.
Unfortunately,
in many developing countries,
insurance agencies
ask their insures
not to move the car
until an adjuster arrives.

Nonetheless,
there are many related location problems
that look at similar issues.
This is the case of
ambulance location problems,
for instance,
where emergency service (ambulance)
must be located
in such a way that
ambulance arrive promptly
at the site of the accident.
A huge difference
between our problem
ant that
of an emergency service location
is that in these problems
it is matter of life-or-death,
whereas in our problem,
is not necessarily so.
This makes
the use of different objective function
and restrictions.
%differ from our practical problem,
%for the next factors
%type of space, %What means this?
%radius of coverage,
%multi-period,
%service time,
%type of arrivals

%Berman %add cite
%proposed a location problem
%for emergency systems,
%and calls The Queue Stochastic p-Median Problem.
In chapter \ref{ch:RelatedWork},
related problems are discussed.

\section{Motivation}
When a car accident occurs,
traffic congestion starts to pile up.
This is because
customers are not allowed to move their vehicles until the adjuster arrives.
The adjuster must record and determine the causes of the accident,
in order to move the car from the accident area
and restore the flow.

\section{Objectives}
The main objective
of this research are:
\begin{itemize}
\item Provide quantitative tools
  for scientific support
  for optimal location
  of car wreck adjusters
\item Develop or use
  adequate mathematical models
  for representing
  some of the important company concerns
\item Design and develop efficient solution techniques
  for handling real-world instance of the problem
\item Assess 
  the quality and value of the proposed techniques
  based on an appropriate experimental design
\end{itemize}
to determine the optimal bases (locations)
for placing the insurance company adjusters,
so as to minimize
the average or maximum response time
from customer calls
when accidents occur.

\section{Organization}
This thesis is organized as follows:
Chapter \ref{ch:RelatedWork}
presents a brief literature review
of divers approaches of Emergency Service Systems (ESS),
starting with the simple deterministic models
and ending with the hypercube model and simulation models.
The problem statement
and two proposed models are presented in Chapter \ref{ch:Framework}.
Chapter \ref{ch:Heuristic} describes the proposed heuristic
and its components.
Chapter \ref{ch:Experiments} contains
the computational test
carried out
with the mathematical models
and the proposed heuristic.
Conclusions, contributions, 
and directions for future research
are highlighted in Chapter \ref{ch:Conclusions}.
Additionally, Appendix \ref{ch:Jarvis}
contains a summary of how to approximate the hypercube results,
and together with Appendix \ref{ch:Berman}
completes the explanation of how to estimate
the response times of a solution.
%cedure discussed in Chapter 5
%Chapter 6 describes
%Chapter 7 contains a
%Conclusions, contributions, 
%and directions for future research are highlighted in Chapter 8
%Additionally, Appendix A


\chapter{Related Work}
In this chapter
the first works
related to location of vehicles
are introduced.
% source: brotcorner2003ambulance
The models are classified in two main categories,
deterministic and probabilistic.
Deterministic models
are used at the planning stage
and ignore stochastic considerations
regarding the availability of vehicles.
Probabilistic models
reflect the fact that
vehicles operate as servers in a queueing system
and cannot always answer a call.
% eos

\section{Covering models}
% source: brotcorner2003ambulance
the location set covering model (LSCM)
introduced Toregas et al. \cite{toregas1971location}
\todo[size=\tiny,caption={Add ref}]{include toregas1971location in refs}
the aim
is to minimize
the number of ambulances needed
to cover all demand points.
This model
ignores several aspects of real-life problems,
the most important
probably being that
once an ambulance is dispatched,
some demand points
are no longer covered.
However,
provide
a lower bound
on the number of ambulances
required to ensure full coverage.

In the maximal covering location problem (MCLP)
originally proposed by Church and ReVelle \cite{church1974maximal}
\todo[size=\tiny,caption={Add ref}]{include church1974maximal in refs}
is an alternative approach
proposed to counter some of the shortcomings of the LSCM
is to maximize population coverage
subject to limited ambulance availability.
% eos

\section{Hypercube}
% source: galvao2008emergency
The hypercube model
is not an optimization model; 
it is only a descriptive model
that permits the analysis of scenarios

Given a system configuration,
the hypercube model
is able to evaluate
a variety of performance measures relevant for decision-making,
either region-wide or for each server or region. 
These include
server workloads,
mean user response times,
fraction of dispatches of each server to each region,
among others.

\subsection{Calibration Process}
% source: galvao2008emergency
% Calibration process of the mean service times
In certain EMSs
and other emergency systems,
travel times may represent
a considerable part of service times.
In such cases,
it may be advisable
to adjust the service times
by means of a calibration process,
which can be performed
using a simple iterative procedure.
Basically,
the procedure consists of
verifying
if there are
significant differences among the input mean service times
and the output mean service times (computed by the hypercube model).
In this case,
the hypercube is solved
using the computed mean service times as inputs,
until
the differences
among input and output values
are sufficiently small.
This procedure is called
a calibration process.
Note that
it takes into account
that the mean travel time
depends on the location of the user
and the identity of the server.
Empirical experiments show
that this procedure
usually converges in two or three iterations,
for a reasonably accurate estimation of the mean service times,
although a formal proof of the convergence of the method
is apparently not available in the literature.

\section{Probabilistic location models}
% source: galvao2008emergency
Probabilistic location models
are optimization models
that permit a more accurate planning
of EMSs at the strategic level.
They were initially developed
based on simplifying assumptions of independence
and non-cooperation among servers,
which are seldom encountered in practice.
% eos
% source: galvao2008emergency
The location of facilities (servers) in ESSs
has as its main objective
the provision of coverage to demand areas.
The definition of
probabilistic location models for planning these systems
was a natural extension
of their deterministic equivalents,
the location models with covering constraints.
The notion of coverage
implies
the definition of a service distance (time),
which is the critical distance (time)
beyond which a demand area is considered not covered.
A demand area is
therefore
considered covered
if
it is within a predefined critical distance (say D)
from at least one of the existing facilities.
%The simplest deterministic covering model
%corresponds to the Location Set Covering Problem (LSCP),
%which provides coverage for every demand area under consideration.
%The provision of total coverage, however,
%may prove to be economically infeasible,
%in the sense that the number of servers required
%may not be compatible with the resources available to the decision-maker.
%The Maximal Covering Location Problem (MCLP)
%was defined by Church and ReVelle (1974) within this context.
%In this case the objective is to locate a number of facilities
%(say p facilities) that is compatible with the resources available,
%such that the maximal possible population of a given geographical region
% is covered within D.

A limitation of the deterministic models
is that
they assume
that servers are available when requested,
which is not always true in practical situations.
Congestion in emergency services,
which may cause
the unavailability
of servers located
within the critical distance
when a call is placed,
lead to
the development of
a second generation of location covering models
focused on additional coverage.
These models
emphasize the importance of additional coverage
for the demand areas,
given the possibility that
in congested systems
the first server,
possibly
the only server in a particular coverage area,
might not be available when requested.
Several such models were developed,
as for example in Daskin and Stern (1981),
Eaton et al. (1981),
Hogan and ReVelle (1986)
and Batta and Mannur (1990).
Probabilistic covering models
are a natural extension
of the second-generation models.
The initial problems
defined within this context
were
the Maximum Expected Covering Location Problem (MEXCLP)
and
the Maximum Availability Location Problem (MALP).
%Extensions of these problems
%lead to the use
%of the hypercube queuing model
%in corresponding solution methodologies.
% eos

\todo[inline]{Include
-Ambulance location
-Cooperative service
-Stocatic Queue p-Median model}

\chapter{Framework}
\label{ch:Framework}
\section{Problem Statement}

% Ideal Situation
The location and dispatching policies
used by insurance agencies
should aim
to arrive to the accident areas
as early as possible,
due to several reasons
such as:
\begin{enumerate}
\item providing a timely service to their customers
\item helping clear out the accident area
\item keeping the workloads of its adjusters
  as balanced as possible
\end{enumerate}
The location policies
should be optimized for service time
but they should also consider
cooperation based on adjuster workload.
% Problem
However,
current insurance agency policies
are empirical,
\todo[inline]{Explicar por que las politicas son empiricas}
and do not consider cooperation.
By neglecting this,
adjusters tend to take
more time to arrive at the accident site,
making the insurance agency
less competitive,
and generating more traffic congestion.

% Financial Costs
The financial costs
of applying
an empirical policies
instead of optimum policies
is difficult to measure.
However the costs include
more use of fuel,
more use of vehicles
that implies
more maintenance costs,
and the opportunity cost
of losing a customer
for low quality-of-service.
% Evidence
The use of quantitative models
may also help
in what-if analysis
to assess the overall service rate
if more adjusters are placed.
% Solution
The use of mathematical models
to determine better location policies
based on scenarios,
and the use of real data
to simulate and evaluate new scenarios
versus the current policy
is one of the main contribution
on this work.

%  Benefits
The benefit of this framework
is that several policies
can be evaluated for different scenarios
(for instance high level of congestion in rainy days),
and the best policy for each scenario can be determined.

% Summary
In summary
the location of adjusters
could be improved
with the use of
mathematical models and simulation,
and obtain several benefits.

% Thesis statement
The problem studied in this thesis
consist of
given a number of adjusters,
a set of potential site for place them (\textbf{basis}),
and a set of demand points,
we have to determine
where to place the adjusters,
so as to minimize 
the average response time,
assuming
that calls
arrive with a Poisson distribution
and with an own arrival rate
for each demand point.
% Conceptual Problems

\section{Mathematical Framework}

Two mathematical models are proposed,
the first model (model A) was created
base on the one proposed by Goldberg \cite{goldberg1990validating}
for which
we made some relaxations
instead for obtaining a linear model.

The second model (model B)
has additional simplification,
considering that
it is unlikely
to assign an adjuster
to a demand point
covered previously
by more adjusters,
omitting allocation variables
and adding restrictions
to guarantee
the correct order of allocation is introduced.

% Validating and applying a model 
% for locating emergency medical vehicles in Tucson, AZ
% Goldberg 1990

\section{Model A}

This model is based
on the model
proposed by Goldberg \cite{goldberg1990validating},
and it contains assignment variables
for all possible orders.

The following assumptions are considered in the model:
\begin{itemize}
\item The probability that an adjuster is busy
  is $\rho$ and it is independent of the state of the system.
\item There is a strict ordering of the basis preferred for each zone
  that does not depend
  on the current state of the system.
\item All calls are answered
  by an adjuster originating from its base,
  not in route back to the base.
\item The arrival of calls to the system
  follows a stationary distribution.
\item The model is presented
  using a 0-queue assumption.
\end{itemize}

Sets and indexes:
\begin{itemize}
\item \textit{n} number of demand points
\item \textit{m} number of potential sites to locate adjusters/facilities
\item \textit{p} number of available adjusters
\item \textit{i} index for demand points;
  $i \in V = \{1,2,\ldots,n\}$ 
\item \textit{j} index for potential sites for adjusters/facilities
  $j \in W = \{1,2,\ldots,m\}$
\item $k$ index for possible order;
  $k \in K = \{1,2\ldots,p\}$
\item $S_{ij} = \{
  r \in W | \mbox{ site } r \mbox{ is preferred by proximity before site } j
  \mbox{ for demand point } i
  \}$
\end{itemize}

Parameters:
\begin{itemize}
\item $\lambda_i$ arrival rate of calls
  from demand point \textit{i}
\item $\rho$ is the utilization of each adjuster,
  the value is between 0 and 1, 
  where 0 means that the server is always idle.
  To obtain an approximate value of $\rho$
  we use the formula proposed by Berman et al.
  \cite{berman1982median}
  \begin{*equation}
    \rho = \frac{\sum_{i=0}^{n}{\lambda_i}}{m p}
  \end{*equation}
\item $t_{ij}$ is the expected travel time
  between demand point \textit{i}
  and potential site \textit{j}.
\item $h_{ij}^{k}$ is the probability
  that adjuster $j$ serves point $i$
  given that
  it is the $k$-th preferred.
  It is calculated
  using the following formula:
  \begin{*equation}
    {h}_{ij}^{k} = (1-\rho)\rho^{k-1}
  \end{*equation}
\end{itemize}

Variables:
\begin{itemize}
\item $x_j =
  \begin{cases} 
    1 & \mbox{if an adjuster is placed at potential site } j \\
    0 & \mbox{otherwise.}
  \end{cases}$
\item $y_{ij}^{k} =
  \begin{cases} 
    1 & \mbox{if the adjuster at site } j \mbox{ is the }
    k\mbox{-th to cover demand point } i \\
    0 & \mbox{otherwise.}
  \end{cases}$
\end{itemize}

Model:
{\small
  \begin{equation}
    \min \, \sum_{j=1}^{m}{
      \sum_{k=1}^{p}{
        \sum_{i=1}^{n}{
          h_{ij}^{k}t_{ij}y_{ij}^{k}
        }
      }
    }
  \end{equation}
}
Minimize the average expected response time
subject to
\begin{align}
  \label{eq:2}
  \sum_{j \in W}{x_j}
  & = p
\end{align}
Locate $p$ adjusters only
\begin{align}
  \label{eq:3}
  \sum_{j \in W}{y_{ij}^{k}}
  & = 1
  & i \in V, k
  &\in K
\end{align}
Each demand point $i$ is covered by an adjuster on each order $k$
\begin{align}
  \label{eq:4}
  y_{ij}^{k}
  & \leq x_j
  & i \in V,j \in W, k
  &\in K
\end{align}
Relationship between variables \textit{x} and \textit{y}
\begin{align}
  \label{eq:5}
  \sum_{k = 1}^{p}{
    y_{ij}^{k}
  }
  & \leq x_j
  & i \in V, j 
  & \in W
\end{align}
For each located adjuster,
there can only be
a maximum of one ordered assignment.
\begin{align}
  y_{ij}^{k} 
  & \leq \sum_{r\in S_{ij}}{y_{ir}^{k-1}}
  & i \in V,j \in W, k
  & \in K\setminus\{1\}
\end{align}
Assign \textit{j} to cover \textit{i} in order $k$
only if
the assignment of order $k-1$
was made for some $r \in S_{ij}$
\begin{align}
  x_{j}
  & \in \{0,1\}
  & j 
  & \in W \nonumber
  \\
  y_{ij}^{k}
  & \in \{0,1\}
  &  i \in V,j \in W,k
  &\in K \nonumber
\end{align}

% Model implications
Observe that
we do not need
to add a constraint
to ensure
the counterpart of (\ref{eq:5})
because (\ref{eq:2}) and (\ref{eq:4})
ensure that
each adjuster
must cover each demand point
for some order,
therefore
if an adjuster located at \textit{j}
does not cover demand point \textit{i}
at order \textit{k}
(indicated by
the maximum covering order in $S_{ij}$)
there will be at least
one adjuster
that does not cover
demand point \textit{i}
at any order
resulting in an infeasible solution.

% A model for the Stochastic Queue p-Median Problem
% Created by Luis Maltos & Roger Rios
% 2015

\section{Model B}

This model
was developed with the idea
that is unlikely that the farthest adjusters
serve demand points
on cases where the system does not become congested.
In these cases
we can make the assumption that
the probability of being served by the $\ell$-th 
adjuster is almost zero,
where $\ell$ is large enough but less than $p$.

Parameters:
\begin{itemize}
\item $M$ is a large integer
\item $\ell$ the number of allowed adjusters per demand point
\item $a_{ik}$ the $k$-th preferred location server
  regarding the point $i$.
\end{itemize}

Variables:
\begin{itemize}
\item $z_j$ the number of adjusters placed at site $j$
\item $y_{ij}^k$ if adjuster in $j$,
  is the $k$-th to cover demand point $i$
\end{itemize}

The objective,
and the constraints \ref{eq:2}-\ref{eq:5} 
are practically the same as in model A,
with the difference that
binary variables $x_j$ from Model A
are replaced by integer variables $z_j$
inspired by the results of Berman \cite{berman1987stochastic},
and the addition of the following binary variables
\begin{itemize}
\item $u_{ij} = 
  \begin{cases}
    1 & \mbox{if the number of adjusters between } i 
    \mbox{ and } j \mbox{, inclusive, is less than } \ell \\
    0 & \mbox{otherwise}
  \end{cases}$
\item $v_{ij} = 
  \begin{cases} 
    1 & \mbox{if the number of adjusters between } i
    \mbox{ and } j \mbox{, is less than } \ell - 1 \\
    0 & \mbox{otherwise}
  \end{cases}$
\end{itemize}

Model
\begin{equation}
  \min \, \sum_{j=1}^{m}{
    \sum_{k=1}^{\ell}{
      \sum_{i=1}^{n}{
        h_{ij}^{k}t_{ij}y_{ij}^{k}
      }
    }
  }
\end{equation}
Minimize the average expected response time
\begin{align}
  \sum_{j \in W}{z_j}
  & = p
\end{align}
Only locate $p$ adjusters
\begin{align}
  \sum_{j \in W}{y_{ij}^{k}}
  & = 1
  & i \in V, k
  &\in \{1,\ldots,\ell\}
\end{align}
Each demand point $i$ is covered by an adjuster on each order until $\ell$
\begin{align}
  y_{ij}^{k}
  & \leq z_j
  & i \in V,j \in W, k
  &\in \{1,\ldots,\ell\}
\end{align}
Relation ship between variables \textit{z} and \textit{y}
\begin{align}
  \label{eq:relzu1}
  \sum_{r \in S_{ij}\cup\{j\}}{
    z_{r}
  }
  + (p-\ell) u_{ij}
  & \leq p
  & i \in V,
  & j \in W 
  \\
  \label{eq:relzu2}
  \sum_{r \in S_{ij}\cup\{j\}}{
    z_{r}
  }
  + M u_{ij}
  & \geq \ell+1
  & i \in V,
  & j \in W
\end{align}
These two constraints
set the relationship
between the $z$ and $u$ variables.
If $u = 1$
the equation (\ref{eq:relzu2}) becomes redundant,
and equation (\ref{eq:relzu1}) guarantees
that the number of adjusters between $i$ and $j$
is less or equal than $\ell$,
otherwise
if $u = 0$
the equation (\ref{eq:relzu1}) becomes redundant,
and equation (\ref{eq:relzu2}) guarantees
that the number of adjusters between $i$ and $j$
is more than $\ell$.
\begin{align}
  \sum_{k = 1}^{\ell}{
    y_{ij}^{k}
  }
  + M (1 - u_{ij})
  & \geq z_j
  & i \in V,
  & j \in W
\end{align}
Assign $z_j$ times $j$ to $i$ if $u_{ij}=1$,
otherwise
becomes redunant.
\begin{align}
  \sum_{r \in S_{ij}}{
    z_{r}
  }
  + (p-(\ell-1)) v_{ij}
  & \leq p
  & i \in V,
  & j \in W
  \\
  \sum_{r \in S_{ij}}{
    z_{r}
  }
  + M v_{ij}
  & \geq \ell
  & i \in V,
  & j \in W
\end{align}
Analogus to equations (\ref{eq:relzu1},\ref{eq:relzu2})
these two constraints
set the relationship
between the $z$ and $v$ variables.
\begin{align}
  \sum_{k=1}^{\ell}{
    y_{ij}^{k}
  }
  + M (1 - v_{ij} + u_{ij})
  & \geq \ell
  - \sum_{r \in S{ij}}{
    z_{r}
  } 
  &  i \in V, j 
  & \in W
  \\
  \sum_{k=1}^{\ell}{
    y_{ij}^{k}
  }
  - M (1 - v_{ij} + u_{ij})
  & \leq \ell 
  - \sum_{r \in S_{ij}}{
    z_{r}
  }
  & i \in V,
  & j \in W
\end{align}
Assign $j$ to $i$
the times remaining to complete $\ell$ assignments
\begin{align}
  y_{ij}^{k}
  & \leq u_{ij}
  + v_{ij}
  & i \in V,
  & j \in W
\end{align}
Assign $j$ to $i$
only if
is in the first $\ell$ adjusters near $i$
\begin{align}
  z_j
  & \in \{0,1,\ldots,p\}
  & j
  & \in V \nonumber
  \\
  y_{ij}^{k} 
  & \in \{0,1\}
  & i\in V,j\in W,k
  & \in I \nonumber
  \\
  u_{ij},v_{ij}
  & \in \{0,1\}
  & i \in V,j
  & \in W \nonumber
\end{align}

\section{Comparison}
As you can see,
the model A
has a considerable amount of binary variables,
it is why the model B was formulated
but due the lack of allocation variables
we need to add other variables and constraints
to ensure a similar behavior.

We can determine the size of the models
in function of \textit{N}, \textit{M}, \textit{p} and $\ell$.

\begin{table}[h]
  \centering
  \begin{tabular}{c|c|c|}
    \cline{2-3}
    & Model A & Model B \\ \hline
    \multicolumn{1}{|l|}{variables} &
    $m(np+1)$ &
    $m(n(\ell+2)+1)$ \\ \hline
    \multicolumn{1}{|l|}{constraints} &
    $n(2mp+p)+1$ &
    $n((l+8)m+1)+1$ \\ \hline
  \end{tabular}
  \caption{Models size}
\end{table}

And we can notice
that since $\ell < p - 2$
have fewer variables
in Model B,
and when $\ell < 2p - 8$
we obtain less constrains
than in Model A.


\chapter{Heuristic Procedures}
\section{GRASP}
%Why did you choose GRASP?
Because is easy,
in the proposed problem
any allocation of sites for adjusters
is a feasible solution,
so it was decided to give some intelligence
to this allocation.
%What does it consist?
We start with a partial solution
(location of a smaller number of adjusters)
with only one adjuster allocated,
evaluate each site with the greedy function,
and choose one from the best $\alpha$ evaluations,
until locate each adjuster.
%Which evaluation functions did you use?
We use three greedy functions
to test the construction.
%Fist function
The first function,
was the p-mean
or the sum the distances
of each allocation 
from each located adjuster
with their nearest demand points.
%Second function
For the second function,
we try to incorporate cooperatively,
including in the evaluation
the distance of allocations
from each located adjuster,
with their $k$ nearest demand points
plus \textit{idle} probability
given that is allocated
to their $k-1$ nearest demand points.
%Third function
The third function,
consist in use the MST calibration method
proposed by Jarvis,
to obtain more accuracy values
of the current mean response time.

\section{Scatter Search}
Components of Scatter Search template \cite{glover1998template}
consist of
specific subroutines of the following types:
\begin{itemize}
\item \textbf{A Diversification Generator:}
  to generate
  a collection of diverse trial solutions,
  using an arbitrary trial solution
  (or seed solution) as an input.
  \begin{algorithm}
  \caption{Path Relinking Initial Phase}\label{pr_init}
  \begin{algorithmic}[0]
    \Procedure{PathRelinking}{$Instance$}
    \State $EliteSols \gets MultiStart(Instance)$
    \Repeat
    \State $MiscSols \gets GenerateMiscSols(Instance,EliteSols)$
    \ForAll{$x \in MiscSols$}
    \State $ImprovementMethod(x)$
    \State $EliteSols.Update(x)$
    \EndFor
    \Until{$PerfectMatchingCost(MiscSols,EliteSols) < \epsilon$}
    \State $SubsetControl(EliteSols)$
    \EndProcedure
  \end{algorithmic}
\end{algorithm}

\item \textbf{An Improvement Method:}
  to transform
  a trial solution
  into one or more enhanced trial solutions. 
  (
  If no improvement
  of the input trial solution results, 
  the ``enhanced'' solution
  is considered to be
  the same as the input solution.)
\item \textbf{A Reference Set Update Method:}
  to build and maintain a Reference Set
  consisting of the \textit{b} best solutions found
  (where the value of \textit{b}
  is typically small,
  e.g., between 20 and 40),
  organized to provide efficient accessing
  by other parts of the method.
  \begin{algorithm}
  \caption{Path Relinking Phase}\label{pr_phase}
  \begin{algorithmic}[0]
    \Procedure{SubsetControl}{$RefSet$}
    \State $NowTime \gets 0$
    \While{$!StopCondition$}
    \State $iNew \gets NumberOfNewSols(RefSet,NowTime)$
    \If{$iNew = 0$}
    \State $exit$
    \EndIf
    \State $GenerateSubsets(RefSet,NowTime)$
    \State $NowTime \gets NowTime + 1$
    \EndWhile
    \EndProcedure
  \end{algorithmic}
\end{algorithm}

\item \textbf{A Subset Generation Method:}
  to operate
  on the Reference Set,
  to produce
  a subset of its solutions
  as a basis
  for creating
  combined solutions.
  \begin{algorithm}
  \caption{Subsets Generator}\label{ss_subsets}
  \begin{algorithmic}[0]
    \Procedure{GenerateSubsets}{$RefSet,NowTime$}
    \State $NewSols \gets SolutionsSince(RefSet,NowTime)$
    \ForAll{$(Sol_x,Sol_y) \in NewSols$}
    \If{$Sol_x \in RefSet$ \textbf{and} $Sol_y \in RefSet$}
    \State $CombinedSols \gets PathRelinkingCombination(Sol_x,Sol_y)$
    \State $Update(RefSet,CombinedSols)$
    \EndIf \EndFor
    \State $UpdateDiversity(RefSet)$
    \If{$NumberOfOldSols(RefSet,NowTime) > 0$}
    \State $OldSols \gets SolutionsUntil(RefSet,NowTime)$
    \ForAll{$Sol_x \in NewSols$}
    \ForAll{$Sol_y \in OldSols$}
    \If{$Sol_x \in RefSet$ \textbf{and} $Sol_y \in RefSet$}
    \State $CombinedSols \gets PathRelinkingCombination(Sol_x,Sol_y)$
    \State $Update(RefSet,CombinedSols)$
    \EndIf \EndFor
    \State $UpdateDiversity(RefSet)$
    \EndFor \EndIf
    \EndProcedure
  \end{algorithmic}
\end{algorithm}

\item \textbf{A Solution Combination Method:}
  to transform
  a given subset of solutions 
  produced by the Subset Generation Method
  into
  one or more combined solution vectors.
  \begin{algorithm}
  \caption{Path Relinking Combination Method}\label{alg:pr_combination}
  \begin{algorithmic}[0]
    \Procedure{PahtRelinkingCombination}{$Sol_x,Sol_y$}
    \State $CombinedSols \gets EmptyList()$
    \State $match \gets Matching(Sol_x,Sol_y)$
    \Comment{perfect,workload,random}
    \State $order \gets ProcessOrder(Sol_x,match,Sol_y)$
    \Comment{nearest/farthest first,random}
    \For{$i \gets 1,p$}
    \State $j \gets order[i]$
    \If{$Sol_x.ServerLocation(j) != Sol_y.ServerLocation(match[j])$}
    \State $Sol_x.SetServerLocation(j,Sol_y.ServerLocation(match[j]))$
    \State $CombinedSols.insert(Sol_x)$
    \EndIf \EndFor
%    \State \textbf{return} $CombinedSols$
    \EndProcedure
  \end{algorithmic}
\end{algorithm}

\end{itemize}

\subsection{Initial Phase}
\begin{enumerate}
\item \textit{Seed Solution Step.}
  Create one or more seed solutions,
  which are
  arbitrary trial solutions
  used to
  initiate
  the remainder of the method.
\item \textit{Diversification Generator.}
  Use the Diversification Generator
  to generate
  diverse trial solutions
  from the seed solution(s).
\item \textit{Improvement and Reference Set Update Methods.}
  For each trial solution
  produced in Step 2,
  use the Improvement Method
  to create one or more enhanced trial solutions.
  During successive applications of this step,
  maintain and update
  a Reference Set
  consisting of the \textit{b} best solutions found.
\item \textit{Repeat.}
  Execute Steps 2 and 3
  until
  producing some designated
  total number of enhanced trial solutions
  as a source of candidates
  for the Reference Set.
\end{enumerate}

\input{HeuristicProcedures/SS_Results}


\chapter{Evaluation}
\chapter{Models}

% Validating and applying a model 
% for locating emergency medical vehicles in Tucson, AZ
% Goldberg 1990

\section{Model A}

This model is based
on the model
proposed by Goldberg \cite{goldberg1990validating},
and it contains assignment variables
for all possible orders.

The following assumptions are considered in the model:
\begin{itemize}
\item The probability that an adjuster is busy
  is $\rho$ and it is independent of the state of the system.
\item There is a strict ordering of the basis preferred for each zone
  that does not depend
  on the current state of the system.
\item All calls are answered
  by an adjuster originating from its base,
  not in route back to the base.
\item The arrival of calls to the system
  follows a stationary distribution.
\item The model is presented
  using a 0-queue assumption.
\end{itemize}

Sets and indexes:
\begin{itemize}
\item \textit{n} number of demand points
\item \textit{m} number of potential sites to locate adjusters/facilities
\item \textit{p} number of available adjusters
\item \textit{i} index for demand points;
  $i \in V = \{1,2,\ldots,n\}$ 
\item \textit{j} index for potential sites for adjusters/facilities
  $j \in W = \{1,2,\ldots,m\}$
\item $k$ index for possible order;
  $k \in K = \{1,2\ldots,p\}$
\item $S_{ij} = \{
  r \in W | \mbox{ site } r \mbox{ is preferred by proximity before site } j
  \mbox{ for demand point } i
  \}$
\end{itemize}

Parameters:
\begin{itemize}
\item $\lambda_i$ arrival rate of calls
  from demand point \textit{i}
\item $\rho$ is the utilization of each adjuster,
  the value is between 0 and 1, 
  where 0 means that the server is always idle.
  To obtain an approximate value of $\rho$
  we use the formula proposed by Berman et al.
  \cite{berman1982median}
  \begin{*equation}
    \rho = \frac{\sum_{i=0}^{n}{\lambda_i}}{m p}
  \end{*equation}
\item $t_{ij}$ is the expected travel time
  between demand point \textit{i}
  and potential site \textit{j}.
\item $h_{ij}^{k}$ is the probability
  that adjuster $j$ serves point $i$
  given that
  it is the $k$-th preferred.
  It is calculated
  using the following formula:
  \begin{*equation}
    {h}_{ij}^{k} = (1-\rho)\rho^{k-1}
  \end{*equation}
\end{itemize}

Variables:
\begin{itemize}
\item $x_j =
  \begin{cases} 
    1 & \mbox{if an adjuster is placed at potential site } j \\
    0 & \mbox{otherwise.}
  \end{cases}$
\item $y_{ij}^{k} =
  \begin{cases} 
    1 & \mbox{if the adjuster at site } j \mbox{ is the }
    k\mbox{-th to cover demand point } i \\
    0 & \mbox{otherwise.}
  \end{cases}$
\end{itemize}

Model:
{\small
  \begin{equation}
    \min \, \sum_{j=1}^{m}{
      \sum_{k=1}^{p}{
        \sum_{i=1}^{n}{
          h_{ij}^{k}t_{ij}y_{ij}^{k}
        }
      }
    }
  \end{equation}
}
Minimize the average expected response time
subject to
\begin{align}
  \label{eq:2}
  \sum_{j \in W}{x_j}
  & = p
\end{align}
Locate $p$ adjusters only
\begin{align}
  \label{eq:3}
  \sum_{j \in W}{y_{ij}^{k}}
  & = 1
  & i \in V, k
  &\in K
\end{align}
Each demand point $i$ is covered by an adjuster on each order $k$
\begin{align}
  \label{eq:4}
  y_{ij}^{k}
  & \leq x_j
  & i \in V,j \in W, k
  &\in K
\end{align}
Relationship between variables \textit{x} and \textit{y}
\begin{align}
  \label{eq:5}
  \sum_{k = 1}^{p}{
    y_{ij}^{k}
  }
  & \leq x_j
  & i \in V, j 
  & \in W
\end{align}
For each located adjuster,
there can only be
a maximum of one ordered assignment.
\begin{align}
  y_{ij}^{k} 
  & \leq \sum_{r\in S_{ij}}{y_{ir}^{k-1}}
  & i \in V,j \in W, k
  & \in K\setminus\{1\}
\end{align}
Assign \textit{j} to cover \textit{i} in order $k$
only if
the assignment of order $k-1$
was made for some $r \in S_{ij}$
\begin{align}
  x_{j}
  & \in \{0,1\}
  & j 
  & \in W \nonumber
  \\
  y_{ij}^{k}
  & \in \{0,1\}
  &  i \in V,j \in W,k
  &\in K \nonumber
\end{align}

% Model implications
Observe that
we do not need
to add a constraint
to ensure
the counterpart of (\ref{eq:5})
because (\ref{eq:2}) and (\ref{eq:4})
ensure that
each adjuster
must cover each demand point
for some order,
therefore
if an adjuster located at \textit{j}
does not cover demand point \textit{i}
at order \textit{k}
(indicated by
the maximum covering order in $S_{ij}$)
there will be at least
one adjuster
that does not cover
demand point \textit{i}
at any order
resulting in an infeasible solution.

% A model for the Stochastic Queue p-Median Problem
% Created by Luis Maltos & Roger Rios
% 2015

\section{Model B}

This model
was developed with the idea
that is unlikely that the farthest adjusters
serve demand points
on cases where the system does not become congested.
In these cases
we can make the assumption that
the probability of being served by the $\ell$-th 
adjuster is almost zero,
where $\ell$ is large enough but less than $p$.

Parameters:
\begin{itemize}
\item $M$ is a large integer
\item $\ell$ the number of allowed adjusters per demand point
\item $a_{ik}$ the $k$-th preferred location server
  regarding the point $i$.
\end{itemize}

Variables:
\begin{itemize}
\item $z_j$ the number of adjusters placed at site $j$
\item $y_{ij}^k$ if adjuster in $j$,
  is the $k$-th to cover demand point $i$
\end{itemize}

The objective,
and the constraints \ref{eq:2}-\ref{eq:5} 
are practically the same as in model A,
with the difference that
binary variables $x_j$ from Model A
are replaced by integer variables $z_j$
inspired by the results of Berman \cite{berman1987stochastic},
and the addition of the following binary variables
\begin{itemize}
\item $u_{ij} = 
  \begin{cases}
    1 & \mbox{if the number of adjusters between } i 
    \mbox{ and } j \mbox{, inclusive, is less than } \ell \\
    0 & \mbox{otherwise}
  \end{cases}$
\item $v_{ij} = 
  \begin{cases} 
    1 & \mbox{if the number of adjusters between } i
    \mbox{ and } j \mbox{, is less than } \ell - 1 \\
    0 & \mbox{otherwise}
  \end{cases}$
\end{itemize}

Model
\begin{equation}
  \min \, \sum_{j=1}^{m}{
    \sum_{k=1}^{\ell}{
      \sum_{i=1}^{n}{
        h_{ij}^{k}t_{ij}y_{ij}^{k}
      }
    }
  }
\end{equation}
Minimize the average expected response time
\begin{align}
  \sum_{j \in W}{z_j}
  & = p
\end{align}
Only locate $p$ adjusters
\begin{align}
  \sum_{j \in W}{y_{ij}^{k}}
  & = 1
  & i \in V, k
  &\in \{1,\ldots,\ell\}
\end{align}
Each demand point $i$ is covered by an adjuster on each order until $\ell$
\begin{align}
  y_{ij}^{k}
  & \leq z_j
  & i \in V,j \in W, k
  &\in \{1,\ldots,\ell\}
\end{align}
Relation ship between variables \textit{z} and \textit{y}
\begin{align}
  \label{eq:relzu1}
  \sum_{r \in S_{ij}\cup\{j\}}{
    z_{r}
  }
  + (p-\ell) u_{ij}
  & \leq p
  & i \in V,
  & j \in W 
  \\
  \label{eq:relzu2}
  \sum_{r \in S_{ij}\cup\{j\}}{
    z_{r}
  }
  + M u_{ij}
  & \geq \ell+1
  & i \in V,
  & j \in W
\end{align}
These two constraints
set the relationship
between the $z$ and $u$ variables.
If $u = 1$
the equation (\ref{eq:relzu2}) becomes redundant,
and equation (\ref{eq:relzu1}) guarantees
that the number of adjusters between $i$ and $j$
is less or equal than $\ell$,
otherwise
if $u = 0$
the equation (\ref{eq:relzu1}) becomes redundant,
and equation (\ref{eq:relzu2}) guarantees
that the number of adjusters between $i$ and $j$
is more than $\ell$.
\begin{align}
  \sum_{k = 1}^{\ell}{
    y_{ij}^{k}
  }
  + M (1 - u_{ij})
  & \geq z_j
  & i \in V,
  & j \in W
\end{align}
Assign $z_j$ times $j$ to $i$ if $u_{ij}=1$,
otherwise
becomes redunant.
\begin{align}
  \sum_{r \in S_{ij}}{
    z_{r}
  }
  + (p-(\ell-1)) v_{ij}
  & \leq p
  & i \in V,
  & j \in W
  \\
  \sum_{r \in S_{ij}}{
    z_{r}
  }
  + M v_{ij}
  & \geq \ell
  & i \in V,
  & j \in W
\end{align}
Analogus to equations (\ref{eq:relzu1},\ref{eq:relzu2})
these two constraints
set the relationship
between the $z$ and $v$ variables.
\begin{align}
  \sum_{k=1}^{\ell}{
    y_{ij}^{k}
  }
  + M (1 - v_{ij} + u_{ij})
  & \geq \ell
  - \sum_{r \in S{ij}}{
    z_{r}
  } 
  &  i \in V, j 
  & \in W
  \\
  \sum_{k=1}^{\ell}{
    y_{ij}^{k}
  }
  - M (1 - v_{ij} + u_{ij})
  & \leq \ell 
  - \sum_{r \in S_{ij}}{
    z_{r}
  }
  & i \in V,
  & j \in W
\end{align}
Assign $j$ to $i$
the times remaining to complete $\ell$ assignments
\begin{align}
  y_{ij}^{k}
  & \leq u_{ij}
  + v_{ij}
  & i \in V,
  & j \in W
\end{align}
Assign $j$ to $i$
only if
is in the first $\ell$ adjusters near $i$
\begin{align}
  z_j
  & \in \{0,1,\ldots,p\}
  & j
  & \in V \nonumber
  \\
  y_{ij}^{k} 
  & \in \{0,1\}
  & i\in V,j\in W,k
  & \in I \nonumber
  \\
  u_{ij},v_{ij}
  & \in \{0,1\}
  & i \in V,j
  & \in W \nonumber
\end{align}

\todo[inline]{Include comments about the complexity of the models}

\section{GRASP}
Several instances were tested
with different $\alpha$ values
to determine the value of $\alpha$ to use.
The value of $\alpha = 0$ means that
the procedure becomes completely greedy,
in contravention
a value of $\alpha = 1$ means that
the procedure becomes completely random.
\missingfigure[figwidth=6cm]{GRASP results of Objective versus $\alpha$}

The test suggest
that
with a more random version
there are more different solutions
and a wide range of objective values
achieving better solutions.

\section{Scatter Search}
To evaluate this heuristic
several instances was tested
and
a desing of experiment was made
to determine
the match method
and the processing order method
that produce better results.

\begin{center}
  \begin{tabular}{|l|l|l|r|r|r|}
    \hline
    M & N & p & Improvement & Time (sec) & MST(sec) \\ \hline
    50 & 30 & 7 & 6.97 & 1.55 & 1.64 \\
    &  & 10 & 9.62 & 5.20 & 4.57 \\
    &  & 15 & 6.95 & 8.55 & 6.49 \\
    &  & 20 & 3.67 & 13.79 & 8.80 \\ \cline{2-6}
    & 50 & 7 & 9.76 & 2.47 & 2.41 \\
    &  & 10 & 12.20 & 7.09 & 6.15 \\
    &  & 15 & 9.75 & 11.35 & 8.51 \\
    &  & 20 & 10.63 & 30.22 & 19.38 \\ \cline{2-6}
    & 75 & 7 & 11.64 & 2.43 & 2.37 \\
    &  & 10 & 14.50 & 6.16 & 5.35 \\
    &  & 15 & 18.26 & 19.44 & 14.08 \\
    &  & 20 & 13.59 & 44.19 & 28.05 \\ \hline
    Total Result &  &  & 10.63 & 12.71 & 8.98 \\ 
    \hline
  \end{tabular}
\end{center}

\begin{center}
  \begin{tabular}{|l|l|l|r|r|r|}
    \hline
    M & N & p & Improvement & Time (sec) & MST(sec) \\ \hline
    100 & 30 & 7 & 6.70 & 4.10 & 4.33 \\
    &  & 10 & 5.74 & 9.97 & 9.65 \\
    &  & 15 & 3.38 & 14.90 & 13.46 \\
    &  & 20 & 2.80 & 28.30 & 24.01 \\ \cline{2-6}
    & 50 & 7 & 7.68 & 4.23 & 4.46 \\
    &  & 10 & 8.09 & 11.01 & 10.59 \\
    &  & 15 & 9.55 & 25.72 & 22.36 \\
    &  & 20 & 9.51 & 61.22 & 49.41 \\ \cline{2-6}
    & 75 & 7 & 7.70 & 4.55 & 4.76 \\
    &  & 10 & 11.46 & 10.17 & 9.82 \\
    &  & 15 & 11.95 & 36.70 & 31.19 \\ \hline
    Total Result &  &  & 7.69 & 19.17 & 16.73 \\ 
    \hline
  \end{tabular}
\end{center}

\missingfigure[figwidth=6cm]{Scatter Search results}


\chapter{Heuristic Procedures}
\section{GRASP}
%Why did you choose GRASP?
Because is easy,
in the proposed problem
any allocation of sites for adjusters
is a feasible solution,
so it was decided to give some intelligence
to this allocation.
%What does it consist?
We start with a partial solution
(location of a smaller number of adjusters)
with only one adjuster allocated,
evaluate each site with the greedy function,
and choose one from the best $\alpha$ evaluations,
until locate each adjuster.
%Which evaluation functions did you use?
We use three greedy functions
to test the construction.
%Fist function
The first function,
was the p-mean
or the sum the distances
of each allocation 
from each located adjuster
with their nearest demand points.
%Second function
For the second function,
we try to incorporate cooperatively,
including in the evaluation
the distance of allocations
from each located adjuster,
with their $k$ nearest demand points
plus \textit{idle} probability
given that is allocated
to their $k-1$ nearest demand points.
%Third function
The third function,
consist in use the MST calibration method
proposed by Jarvis,
to obtain more accuracy values
of the current mean response time.

\section{Scatter Search}
Components of Scatter Search template \cite{glover1998template}
consist of
specific subroutines of the following types:
\begin{itemize}
\item \textbf{A Diversification Generator:}
  to generate
  a collection of diverse trial solutions,
  using an arbitrary trial solution
  (or seed solution) as an input.
  \begin{algorithm}
  \caption{Path Relinking Initial Phase}\label{pr_init}
  \begin{algorithmic}[0]
    \Procedure{PathRelinking}{$Instance$}
    \State $EliteSols \gets MultiStart(Instance)$
    \Repeat
    \State $MiscSols \gets GenerateMiscSols(Instance,EliteSols)$
    \ForAll{$x \in MiscSols$}
    \State $ImprovementMethod(x)$
    \State $EliteSols.Update(x)$
    \EndFor
    \Until{$PerfectMatchingCost(MiscSols,EliteSols) < \epsilon$}
    \State $SubsetControl(EliteSols)$
    \EndProcedure
  \end{algorithmic}
\end{algorithm}

\item \textbf{An Improvement Method:}
  to transform
  a trial solution
  into one or more enhanced trial solutions. 
  (
  If no improvement
  of the input trial solution results, 
  the ``enhanced'' solution
  is considered to be
  the same as the input solution.)
\item \textbf{A Reference Set Update Method:}
  to build and maintain a Reference Set
  consisting of the \textit{b} best solutions found
  (where the value of \textit{b}
  is typically small,
  e.g., between 20 and 40),
  organized to provide efficient accessing
  by other parts of the method.
  \begin{algorithm}
  \caption{Path Relinking Phase}\label{pr_phase}
  \begin{algorithmic}[0]
    \Procedure{SubsetControl}{$RefSet$}
    \State $NowTime \gets 0$
    \While{$!StopCondition$}
    \State $iNew \gets NumberOfNewSols(RefSet,NowTime)$
    \If{$iNew = 0$}
    \State $exit$
    \EndIf
    \State $GenerateSubsets(RefSet,NowTime)$
    \State $NowTime \gets NowTime + 1$
    \EndWhile
    \EndProcedure
  \end{algorithmic}
\end{algorithm}

\item \textbf{A Subset Generation Method:}
  to operate
  on the Reference Set,
  to produce
  a subset of its solutions
  as a basis
  for creating
  combined solutions.
  \begin{algorithm}
  \caption{Subsets Generator}\label{ss_subsets}
  \begin{algorithmic}[0]
    \Procedure{GenerateSubsets}{$RefSet,NowTime$}
    \State $NewSols \gets SolutionsSince(RefSet,NowTime)$
    \ForAll{$(Sol_x,Sol_y) \in NewSols$}
    \If{$Sol_x \in RefSet$ \textbf{and} $Sol_y \in RefSet$}
    \State $CombinedSols \gets PathRelinkingCombination(Sol_x,Sol_y)$
    \State $Update(RefSet,CombinedSols)$
    \EndIf \EndFor
    \State $UpdateDiversity(RefSet)$
    \If{$NumberOfOldSols(RefSet,NowTime) > 0$}
    \State $OldSols \gets SolutionsUntil(RefSet,NowTime)$
    \ForAll{$Sol_x \in NewSols$}
    \ForAll{$Sol_y \in OldSols$}
    \If{$Sol_x \in RefSet$ \textbf{and} $Sol_y \in RefSet$}
    \State $CombinedSols \gets PathRelinkingCombination(Sol_x,Sol_y)$
    \State $Update(RefSet,CombinedSols)$
    \EndIf \EndFor
    \State $UpdateDiversity(RefSet)$
    \EndFor \EndIf
    \EndProcedure
  \end{algorithmic}
\end{algorithm}

\item \textbf{A Solution Combination Method:}
  to transform
  a given subset of solutions 
  produced by the Subset Generation Method
  into
  one or more combined solution vectors.
  \begin{algorithm}
  \caption{Path Relinking Combination Method}\label{alg:pr_combination}
  \begin{algorithmic}[0]
    \Procedure{PahtRelinkingCombination}{$Sol_x,Sol_y$}
    \State $CombinedSols \gets EmptyList()$
    \State $match \gets Matching(Sol_x,Sol_y)$
    \Comment{perfect,workload,random}
    \State $order \gets ProcessOrder(Sol_x,match,Sol_y)$
    \Comment{nearest/farthest first,random}
    \For{$i \gets 1,p$}
    \State $j \gets order[i]$
    \If{$Sol_x.ServerLocation(j) != Sol_y.ServerLocation(match[j])$}
    \State $Sol_x.SetServerLocation(j,Sol_y.ServerLocation(match[j]))$
    \State $CombinedSols.insert(Sol_x)$
    \EndIf \EndFor
%    \State \textbf{return} $CombinedSols$
    \EndProcedure
  \end{algorithmic}
\end{algorithm}

\end{itemize}

\subsection{Initial Phase}
\begin{enumerate}
\item \textit{Seed Solution Step.}
  Create one or more seed solutions,
  which are
  arbitrary trial solutions
  used to
  initiate
  the remainder of the method.
\item \textit{Diversification Generator.}
  Use the Diversification Generator
  to generate
  diverse trial solutions
  from the seed solution(s).
\item \textit{Improvement and Reference Set Update Methods.}
  For each trial solution
  produced in Step 2,
  use the Improvement Method
  to create one or more enhanced trial solutions.
  During successive applications of this step,
  maintain and update
  a Reference Set
  consisting of the \textit{b} best solutions found.
\item \textit{Repeat.}
  Execute Steps 2 and 3
  until
  producing some designated
  total number of enhanced trial solutions
  as a source of candidates
  for the Reference Set.
\end{enumerate}

\input{HeuristicProcedures/SS_Results}



\appendix
%%% Haz un documento para cada apéndice
% Approximating the Equilibrium Behavior of Multi-Server Loss Systems
% Jarvis - Management Sience
% 1985

\chapter{Approximating the Equilibrium Behavior of MSLS}
In Jarvis \cite{jarvis1985approximating}
a procedure is given
for approximating the equilibrium behavior
of multi-server loss systems having distinguishable servers
and multiple customers types
under light to moderate traffic intensity.

\section{Introduction}
In an emergency service such as fire or police, 
the servers are fire fighting units or patrol cars
and the customers are calls for service.
The simple Erlang loss system
is inadequate in two aspects for a detailed system analysis.
\begin{enumerate}
\item one often wishes
  to preserve the identity of service units (distinguishable servers).
\item because
  of the geographic nature of these systems,
  the service time depend on both
  the server and the customer
  at least through the travel time between the pair.
\end{enumerate}

\section{Model assumptions, Notation, and Terminology}
Consider a system in which:
\begin{enumerate}
\item Exactly one server
  is assigned to each customer
  unless all servers are busy, 
  in which case
  the customer is irrevocably lost of the system
\item Servers are assigned
  to customers according to a fixed preference assignment rule
\item No preemption of service is allowed
\item Assignments are made immediately
  upon customer arrival
\end{enumerate}

and we have the following parameters
\begin{enumerate}
\item \textit{N} distinguishable servers
\item \textit{C} types of customers
\item Customers of type \textit{m}
  arrive according to a Poisson process with rate $\lambda_{m}$
\item $\lambda$ total arrival rate
\item $a_{mk}$ be the \textit{k}th preferred server
  for customers of type \textit{m}
\item $\tau_{im}$ the expected service time
  for server \textit{i} and customer of type \textit{m}
\end{enumerate}

The performance measures for the system include 
\begin{enumerate}
\item $\rho_i$ the workload of server \texit{i}
\item $f_{im}$ the probability a random customer of type \textit{m}
  is assigned to server \textit{i}
\item $P_{N}$ the probability all servers are busy
\end{enumerate}

\section{Approximation Procedure}
The procedure described is based on that given by
Larson \cite{larson1975approximating},
for approximating performance measures
for the Hypercube model
assuming exponential service times.
Larson developed an approximation for $f_{im}$ as
\begin{equation} \label{f_im}
  f_{im} \simeq Q(N,p,k-1)(1-\rho_{i})\prod_{l=1}^{k-1}{\rho_{a_{ml}}}
\end{equation}
where
\begin{equation} \label{Q}
  Q(N,p,k) =
  \sum_{j=k}^{N-1}{
    \frac{(N-j)(N^j)(\rho^{j-k})P_0(N-k-1)!}
         {(j-k)!(1-P_N)^kN!(1-\rho(1-P_N))}
  } \\
  \hfill \mbox{for} k = 0,1,\ldots,N-1
\end{equation}

Let $B_i$ denote the event
that server \textit{i} is busy;
and let $B_{im}$ denote the event
that server \textit{i} is busy
serving a customer of type \textit{m}, then
\begin{equation} \label{rho_i}
  \rho_{i} = 
  Pr\left[B_i\right] = 
  \sum_{m=1}^{C}{
    Pr\left[B_{im}\right]} =
  \sum_{m=1}^{C}{
    \lambda_{m}f_{im}\tau_{im}
  }
\end{equation}
combine equations (\ref{f_im}) and (\ref{rho_i})
and solve for $\rho_i$ to obtain the approximation iteration
\begin{equation}
  \rho_i\mbox{(new)} =
  \frac{V_i}
       {(1+V_i)}
\end{equation}
where $V_i$ is given by
\begin{equation} \label{V_i}
  V_i = \sum_{k=1}^{N}{\sum_{m:a_{mk}=i}{\lambda_m \tau_{im} Q(N,\rho,k-1)\prod_{l=1}^{k-1}{\rho_{a_{ml}}}}}
\end{equation}

When there is a common mean service time, the estimates for $\rho_i$ can be normalized using
\begin{equation} \label{P_N}
  \sum_{i=1}^{N}{\rho_i} = N \rho (1 - P_{N})
\end{equation}

In the generalized procedure, $\tau$ can be approximated at the end of each iteration by
\begin{equation} \label{tau}
  \tau = \sum_{m=1}^{C}{\left(\frac{\lambda_m}{\lambda}\right)\sum_{i=1}^{N}{\frac{\tau_{im}f_{im}}{(1-P_N)}}}
\end{equation}

{\footnotesize
  \begin{center}
    \textit{ Approximation Algorithm}
  \end{center}
  \vspace{-8pt}
  \hline \\
  \vspace{2pt}
  \textit{Given:}
  \begin{equation*}
    \lambda_m,\tau_{im},a_{mk}\mbox{  for } m = 1,\ldots,C;i=1,\ldots,N;k=1,\ldots,N \hfill
  \end{equation*}
  \textit{Initialize:}
  \begin{equation*}
    \rho_i = \sum_{m: a_{m1} = i}{\lambda_m \tau_{im}}; \; \tau = \sum_{m=1}^{C}{(\lambda_m/\lambda)\tau_{a_{m1},m}} \hfill
  \end{equation*}
  \textit{Iteration:}\\
  (1) Compute $Q(N,\rho,k)$ for $k = 1,\ldots,N-1$ where $\rho = \lambda \tau / N$ using equation (\ref{Q}). \\
  (2) For $i = 1,\ldots,N$, the new $\rho_i$ is $V_i/(1+V_i)$, where $V_i$ is given by equation (\ref{V_i}). \\
  (3) Stop if max change in $\rho_i$ is less than convergence criterion. \\
  (4) Else compute $P_N$ by equation (\ref{P_N}), $\tau$ by equation (\ref{tau}), and $f_{im}$ by equation (\ref{f_im}). \\
  (5) Return to step 1. \\
  \hline
}

No analytic bounds on the accuracy or convergence properties of the approximation procedure have been developed.

In regards to convergence properties,
the numerical iteration has proved to be very stable and converges in a small number of iterations under relatively stringent conditions,
with 4 to 6 iterations being typical for 10 servers systems.

In comparing the accuracy of this approximation to results of the exact Hypercube model,
Larson has found errors in server workloads to be less than 1 to 2 percent.

% The Stochastic Queue p-Median Problem
% O. Berman, R.C. Larson, C. Parkan 
% 1987
\chapter{Berman Heuristic}
\label{ch:Berman}
This problem is presented by Berman et al. \cite{berman1987stochastic}
and describes a similar situation to our case,
with the difference
that the location of servers
is in all the network,
including the edges.

Also present two easily programmable heuristics
to solve the proposed problem.
And uses the \textbf{Mean Service Time Calibration Process}
to evaluate the solutions.

\section{The Stochastic Queue p-Median Problem}
To define the problem
the following definitions are made
\begin{itemize}
\item $G(N,L)$ the transportation network
\item $N$ the set of demand centers,
  with $|N| = n$
\item $L$ the set of all transportation arteries,
  the links
\item $h_j$ the fraction of service calls
  associated with each node $j$
\item $d(X,Y)$ is the shortest path between any two points $X,Y \in G$
\item $p$ number of response units
\item $\bar{X}$ the home locations of the service units while available
\item $\lambda$ mean rate per unit of time
  within service calls are generated in Poisson manner
\end{itemize}

Given the arrival of a call for service,
exactly one of the servers is dispatched to it
assuming that at least one server is available.

The service time for any service unit \textit{i}
is the sum of two components:
\begin{itemize}
\item The non-travel time component,
  which is the sum of
  on-scene and off-scene service time.
\item Travel time component,
  which is the sum of travel time
  to the location of the call
  and travel time back to the home location.
\end{itemize}

The mean service time 
for a service unit located at $\bar{X}^i$ 
is denoted $S(\bar{X}^i)$,
\begin{equation}
  S(\bar{X}^i) =
  \sum_{j = 1}^{n} {
    h_j^i\left(
    \bar{W}_{ij}
    +\frac{{\beta}_i}{v_i}d(\bar{X}^i,j)
    \right)
  } \; i=1,\ldots,p
\end{equation}
\begin{itemize}
\item $\bar{W}_{ij}$ is the mean
  of the \textbf{non-travel time} component $W_{ij}$
\item $v_i$ is the \textbf{travel speed} of unit \textit{i}
  to the scene of the call
  which is assumed constant
\item $\beta_i$ is a constant
  that allows different travel speeds
  to and from the scene of the call
\item $h_j^i$ is the probability that server \textit{i} 
  is dispatched to node \textit{j}
  given that server \textit{i}
  is dispatched to a call for service.
\end{itemize}

Whenever a call for service arrives 
while at least one of the servers
is free at its home location,
the closest available server to the call
will be dispatched.
Calls that find
all servers busy
enter a queue.
The queue discipline is assumed to be
First-Come-First-Served.
The expected response time
to a random call
denoted by $\bar{T}_R(X)$
is the sum of two components
\begin{equation*}
  \bar{T}_R(\bar{X}) = \bar{W}_q(\bar{X})+\bar{t}(\bar{X})
\end{equation*}
\begin{itemize}
\item $\bar{W}_q(\bar{X})$ is the expected \textbf{waiting time} in the queue
\item $\bar{t}(\bar{X})$ is the expected \textbf{travel time} to the call.
\end{itemize}
The objective is to find
a set of \textit{p} locations $\bar{X}^*$
on the network
such that
\begin{equation*}
  \bar{T}_R(\bar{X}^*) \leq \bar{T}_R(\bar{X}) \; \forall \bar{X} \in G
\end{equation*}
$\bar{X}^*$ is called \textbf{the stochastic queue p-median}.

\section{Mean Service Time Calibration Process}
In this emergency system, 
travel times may represent a considerable part of service times. 
It may be advisable
to adjust the service times by means of a calibration process,
which can be performed using a simple iterative procedure.

The procedure consists
of verifying if there are significant differences
among the input mean service times
and the output mean service times (computed by the hypercube model). 
In this case, 
the hypercube
is solved using the computed mean service times as inputs, 
until the differences among input and output values
are sufficiently small.

\begin{center}
  The mean service time calibration method
\end{center}
\vspace{-8pt}
\hline \\
\vspace{2pt}
\textit{Step 0.} The mean service time of unit \textit{i},
$1/\mu^{i}=1/\mu_{NT}^{i}$, $i = 1,\ldots,p$ 
($1/\mu_{NT}^{i} = \sum_{j=1}^{n}{h_j\bar{W_{ij}}}$ 
is the mean of non-travel time component of the service time). \\
\textit{Step 1.} Run the Hypercube Model (using $\mu^i$) to obtain
$f_{ij}$, $i = 1,\ldots,p$, $j = 1,\ldots,n$. \\
\textit{Step 2.} $1/\hat{\mu}^{i} =
\sum_{k=1}^{n}{
  h_{k}^{i}(\bar{W_{ik}}+(\beta_{i}/v_{i})d_{ik})
}$ 
where 
$h_{k}^{i} = f_{ik}/\sum_{j=1}^{n}{f_{ij}}$ \\
\textit{Step 3.} If $|1/\hat{\mu}^{i}-1/\mu^{i}|>\epsilon$
for at least one $i$, $i=1,\ldots,p$,
set $1/\mu^{i} \equiv 1/\hat{\mu}^{i}$
and go back to \textit{Step 1.}
Otherwise stop. \\
\hline




\backmatter
\pagestyle{main}

%%% Aquí va la bibliografía, puedes usar el entorno de LaTeX (thebibliography)
%%% o la herramienta BibTeX. En caso de que optes por BibTeX, puedes usar
%%% alguno de los archivos de estilo (mighelbib.bst o mighelnat.bst) incluidos
%%% en el paquete, cuyos diseños armonizan con el diseño de tesis provisto por
%%% fime.cls. Para muestra, basta un botón:
\bibliographystyle{abbrv}
\bibliography{references}

\label{lastpage}
\include{Misc/Autobiography}

\end{document}
