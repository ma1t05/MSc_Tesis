% The Stochastic Queue p-Median Problem
% O. Berman, R.C. Larson, C. Parkan 
% 1987
\chapter{Berman Heuristic}
\label{ch:Berman}
This problem is presented by Berman et al. \cite{berman1987stochastic}
and describes a similar situation to our case,
with the difference
that the location of servers
is in all the network,
including the edges.

Also present two easily programmable heuristics
to solve the proposed problem.
And uses the \textbf{Mean Service Time Calibration Process}
to evaluate the solutions.

\section{The Stochastic Queue p-Median Problem}
To define the problem
the following definitions are made
\begin{itemize}
\item $G(N,L)$ the transportation network
\item $N$ the set of demand centers,
  with $|N| = n$
\item $L$ the set of all transportation arteries,
  the links
\item $h_j$ the fraction of service calls
  associated with each node $j$
\item $d(X,Y)$ is the shortest path between any two points $X,Y \in G$
\item $p$ number of response units
\item $\bar{X}$ the home locations of the service units while available
\item $\lambda$ mean rate per unit of time
  within service calls are generated in Poisson manner
\end{itemize}

Given the arrival of a call for service,
exactly one of the servers is dispatched to it
assuming that at least one server is available.

The service time for any service unit \textit{i}
is the sum of two components:
\begin{itemize}
\item The non-travel time component,
  which is the sum of
  on-scene and off-scene service time.
\item Travel time component,
  which is the sum of travel time
  to the location of the call
  and travel time back to the home location.
\end{itemize}

The mean service time 
for a service unit located at $\bar{X}^i$ 
is denoted $S(\bar{X}^i)$,
\begin{equation}
  S(\bar{X}^i) =
  \sum_{j = 1}^{n} {
    h_j^i\left(
    \bar{W}_{ij}
    +\frac{{\beta}_i}{v_i}d(\bar{X}^i,j)
    \right)
  } \; i=1,\ldots,p
\end{equation}
\begin{itemize}
\item $\bar{W}_{ij}$ is the mean
  of the \textbf{non-travel time} component $W_{ij}$
\item $v_i$ is the \textbf{travel speed} of unit \textit{i}
  to the scene of the call
  which is assumed constant
\item $\beta_i$ is a constant
  that allows different travel speeds
  to and from the scene of the call
\item $h_j^i$ is the probability that server \textit{i} 
  is dispatched to node \textit{j}
  given that server \textit{i}
  is dispatched to a call for service.
\end{itemize}

Whenever a call for service arrives 
while at least one of the servers
is free at its home location,
the closest available server to the call
will be dispatched.
Calls that find
all servers busy
enter a queue.
The queue discipline is assumed to be
First-Come-First-Served.
The expected response time
to a random call
denoted by $\bar{T}_R(X)$
is the sum of two components
\begin{equation*}
  \bar{T}_R(\bar{X}) = \bar{W}_q(\bar{X})+\bar{t}(\bar{X})
\end{equation*}
\begin{itemize}
\item $\bar{W}_q(\bar{X})$ is the expected \textbf{waiting time} in the queue
\item $\bar{t}(\bar{X})$ is the expected \textbf{travel time} to the call.
\end{itemize}
The objective is to find
a set of \textit{p} locations $\bar{X}^*$
on the network
such that
\begin{equation*}
  \bar{T}_R(\bar{X}^*) \leq \bar{T}_R(\bar{X}) \; \forall \bar{X} \in G
\end{equation*}
$\bar{X}^*$ is called \textbf{the stochastic queue p-median}.

\section{Mean Service Time Calibration Process}
In this emergency system, 
travel times may represent a considerable part of service times. 
It may be advisable
to adjust the service times by means of a calibration process,
which can be performed using a simple iterative procedure.

The procedure consists
of verifying if there are significant differences
among the input mean service times
and the output mean service times (computed by the hypercube model). 
In this case, 
the hypercube
is solved using the computed mean service times as inputs, 
until the differences among input and output values
are sufficiently small.

\begin{center}
  The mean service time calibration method
\end{center}
\vspace{-8pt}
\hline \\
\vspace{2pt}
\textit{Step 0.} The mean service time of unit \textit{i},
$1/\mu^{i}=1/\mu_{NT}^{i}$, $i = 1,\ldots,p$ 
($1/\mu_{NT}^{i} = \sum_{j=1}^{n}{h_j\bar{W_{ij}}}$ 
is the mean of non-travel time component of the service time). \\
\textit{Step 1.} Run the Hypercube Model (using $\mu^i$) to obtain
$f_{ij}$, $i = 1,\ldots,p$, $j = 1,\ldots,n$. \\
\textit{Step 2.} $1/\hat{\mu}^{i} =
\sum_{k=1}^{n}{
  h_{k}^{i}(\bar{W_{ik}}+(\beta_{i}/v_{i})d_{ik})
}$ 
where 
$h_{k}^{i} = f_{ik}/\sum_{j=1}^{n}{f_{ij}}$ \\
\textit{Step 3.} If $|1/\hat{\mu}^{i}-1/\mu^{i}|>\epsilon$
for at least one $i$, $i=1,\ldots,p$,
set $1/\mu^{i} \equiv 1/\hat{\mu}^{i}$
and go back to \textit{Step 1.}
Otherwise stop. \\
\hline


