% Approximating the Equilibrium Behavior of Multi-Server Loss Systems
% Jarvis - Management Sience
% 1985

\chapter{Approximating the Equilibrium Behavior of MSLS}
In Jarvis \cite{jarvis1985approximating}
a procedure is given
for approximating the equilibrium behavior
of multi-server loss systems having distinguishable servers
and multiple customers types
under light to moderate traffic intensity.

\section{Introduction}
In an emergency service such as fire or police, 
the servers are fire fighting units or patrol cars
and the customers are calls for service.
The simple Erlang loss system
is inadequate in two aspects for a detailed system analysis.
\begin{itemize}
\item one often wishes
  to preserve the identity of service units (distinguishable servers).
\item because
  of the geographic nature of these systems,
  the service time depend on both
  the server and the customer
  at least through the travel time between the pair.
\end{itemize}

\section{Model assumptions, Notation, and Terminology}
Consider a system in which:
\begin{itemize}
\item Exactly one server
  is assigned to each customer
  unless all servers are busy, 
  in which case
  the customer is irrevocably lost from the system
\item Servers are assigned
  to customers according to a fixed preference assignment rule
\item No preemption of service is allowed
\item Assignments are made immediately
  upon customer arrival
\end{itemize}

and we have the following parameters
\begin{itemize}
\item \textit{N} distinguishable servers
\item \textit{C} types of customers
\item Customers of type \textit{m}
  arrive according to a Poisson process with rate $\lambda_{m}$
\item $\lambda$ total arrival rate
\item $a_{mk}$ be the \textit{k}th preferred server
  for customers of type \textit{m}
\item $\tau_{im}$ the expected service time
  for server \textit{i} and customer of type \textit{m}
\end{itemize}

The performance measures for the system include 
\begin{itemize}
\item $\rho_i$ the workload of server \texit{i}
\item $f_{im}$ the probability a random customer of type \textit{m}
  is assigned to server \textit{i}
\item $P_{N}$ the probability all servers are busy
\end{itemize}

\section{Approximation Procedure}
The procedure described below
is based on that given by
Larson \cite{larson1975approximating},
for approximating performance measures
for the Hypercube model
assuming exponential service times.
Larson developed an approximation for $f_{im}$ as
\begin{equation} \label{f_im}
  f_{im} \simeq Q(N,p,k-1)(1-\rho_{i})\prod_{l=1}^{k-1}{\rho_{a_{ml}}}
\end{equation}
where
\begin{equation} \label{Q}
  Q(N,p,k) =
  \sum_{j=k}^{N-1}{
    \frac{(N-j)(N^j)(\rho^{j-k})P_0(N-k-1)!}
         {(j-k)!(1-P_N)^kN!(1-\rho(1-P_N))}
  } \\
  \hfill \mbox{for} k = 0,1,\ldots,N-1
\end{equation}

Let $B_i$ denote the event
that server \textit{i} is busy;
and let $B_{im}$ denote the event
that server \textit{i} is busy
serving a customer of type \textit{m}, then
\begin{equation} \label{rho_i}
  \rho_{i} = 
  Pr\left[B_i\right] = 
  \sum_{m=1}^{C}{
    Pr\left[B_{im}\right]} =
  \sum_{m=1}^{C}{
    \lambda_{m}f_{im}\tau_{im}
  }
\end{equation}
combine equations (\ref{f_im}) and (\ref{rho_i})
and solve for $\rho_i$ to obtain the approximation iteration
\begin{equation}
  \rho_i\mbox{(new)} =
  \frac{V_i}
       {(1+V_i)}
\end{equation}
where $V_i$ is given by
\begin{equation} \label{V_i}
  V_i = \sum_{k=1}^{N}{\sum_{m:a_{mk}=i}{\lambda_m \tau_{im} Q(N,\rho,k-1)\prod_{l=1}^{k-1}{\rho_{a_{ml}}}}}
\end{equation}

When there is a common mean service time, the estimates for $\rho_i$ can be normalized using
\begin{equation} \label{P_N}
  \sum_{i=1}^{N}{\rho_i} = N \rho (1 - P_{N})
\end{equation}

In the generalized procedure, $\tau$ can be approximated at the end of each iteration by
\begin{equation} \label{tau}
  \tau = \sum_{m=1}^{C}{\left(\frac{\lambda_m}{\lambda}\right)\sum_{i=1}^{N}{\frac{\tau_{im}f_{im}}{(1-P_N)}}}
\end{equation}

{\footnotesize
  \begin{center}
    \textit{ Approximation Algorithm}
  \end{center}
  \vspace{-8pt}
  \hline \\
  \vspace{2pt}
  \textit{Given:}
  \begin{equation*}
    \lambda_m,\tau_{im},a_{mk}\mbox{  for } m = 1,\ldots,C;i=1,\ldots,N;k=1,\ldots,N \hfill
  \end{equation*}
  \textit{Initialize:}
  \begin{equation*}
    \rho_i = \sum_{m: a_{m1} = i}{\lambda_m \tau_{im}}; \; \tau = \sum_{m=1}^{C}{(\lambda_m/\lambda)\tau_{a_{m1},m}} \hfill
  \end{equation*}
  \textit{Iteration:}\\
  (1) Compute $Q(N,\rho,k)$ for $k = 1,\ldots,N-1$ where $\rho = \lambda \tau / N$ using equation (\ref{Q}). \\
  (2) For $i = 1,\ldots,N$, the new $\rho_i$ is $V_i/(1+V_i)$, where $V_i$ is given by equation (\ref{V_i}). \\
  (3) Stop if max change in $\rho_i$ is less than convergence criterion. \\
  (4) Else compute $P_N$ by equation (\ref{P_N}), $\tau$ by equation (\ref{tau}), and $f_{im}$ by equation (\ref{f_im}). \\
  (5) Return to step 1. \\
  \hline
}

No analytic bounds
on the accuracy or convergence properties
of the approximation procedure
have been developed
to the best of my knowledge.

In regards to convergence properties,
the numerical iteration
has proved to be very stable
and converges in a small number of iterations
under relatively stringent conditions,
with 4 to 6 iterations
being typical for 10-servers systems.

In comparing the accuracy of this approximation
to results of the exact Hypercube model,
Larson has found errors in server workloads
to be less than 1 to 2 percent.
